\begin{abstract}
Students often have difficulties in learning how to program in an object-oriented style.
Frequently, the programming tools used in initial courses 
are either too complex for the students at that stage, 
or simplified in such a way that makes the transition to industrial work environments hard.
The languages used require them to understand a lot of abstract concepts, 
even for producing simple programs.
These deficiencies can be perceived in later courses or even in professional practice.

In this paper we present the \emph{Wollok IDE}, 
a novel educative development environment to teach object-oriented programming. 
\emph{Wollok IDE} differs from existing teaching tools in that its features and design are conceived to support a carefully designed \emph{incremental learning path}. 
that makes it easy to start with a minimal set of concepts 
and then smoothly transition to more complex models and tools, 
including in particular industrial ones.

We also enumerate the pedagogical motivations that guided the design of Wollok's main features,
and how they support and enhance the learning process.
We have evaluated this tool using it in around 30 teaching courses over 2 years 
including more than 1000 students.
We show data of students interviews and surveys
which reveals a higher ratio of approval and a deeper understanding of theoretical concepts.
\end{abstract}

%This learning path is supported by a customized development environment which enables the creation of programs using this \emph{simplified programming model}, 
%and allows us to postpone the introduction of more abstract concepts like classes or inheritance. 
%
%we propose an enhancement to this learning path focusing on the transitions between the stages of the course. 
%We also present a new educational programming language named Wollok, which allows maximizing the accuracy in the selection of concepts to present. 
%Finally, Wollok is accompanied by a programming environment which has lots of tools to guide the student and to help detecting mistakes
%and at the same time is in line with the most common professional practices.


%Also, educative tools that differ greatly from those used in the IT industry 
%can weaken student interest,
%as well as hamper the application of the learned concepts and techniques 
%in subsequent labor experiences. 
