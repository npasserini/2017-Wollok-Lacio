\begin{abstract}
Students often have difficulties in learning how to program in an object-oriented style. 
One of the causes of this problem is that 
object-oriented languages require the programmer to be familiarized
with a big amount of non-trivial concepts. 
While there exist languages that can reduce this complexity, 
it is difficult to find programming tools for 
a student who is walking her first steps into programming.

In our experience, the lack of adecquate programming languages and tools
can be a significant obstacle for initial courses.
Also, educative tools that differ greatly from those used in the IT industry 
can weaken student interest,
as well as hamper the application of the learned concepts and techniques 
in subsequent labor experiences. 

In this work we describe the \emph{Wollok IDE}, an educative development environment
conceived for learning OOP in a way that supports 
an \emph{incremental learning path}, 
and, at the same time, facilitates the transition to industrial environments.
Equally important, we briefly describe 
the motivations for their design
and our teaching experience with these tools.
\end{abstract}

%This learning path is supported by a customized development environment which enables the creation of programs using this \emph{simplified programming model}, 
%and allows us to postpone the introduction of more abstract concepts like classes or inheritance. 
%
%we propose an enhancement to this learning path focusing on the transitions between the stages of the course. 
%We also present a new educational programming language named Wollok, which allows maximizing the accuracy in the selection of concepts to present. 
%Finally, Wollok is accompanied by a programming environment which has lots of tools to guide the student and to help detecting mistakes
%and at the same time is in line with the most common professional practices.
