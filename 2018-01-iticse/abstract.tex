\begin{abstract}
Students often have difficulties in learning how to program in an object-oriented style.
Frequently, they do not succeed to comprehend all the abstract concepts involved 
or they are not able to apply their learnings in professional practice.
One important reason for these problems is the lack of adecquate programming languages and tools.

In this paper we present the \emph{Wollok IDE}, an educative development environment
that allows for an \emph{incremental learning path},
starting with a minimal set of concepts
and providing smooth transitions to more complex models and tools, 
including industrial ones.
We give a brief summary of the pedagogical view that backs our design desicions
and we describe in which way these tools have been successful 
in improving the learning process and, thus, 
the outcome of object-oriented programming courses at university level.
\end{abstract}

%This learning path is supported by a customized development environment which enables the creation of programs using this \emph{simplified programming model}, 
%and allows us to postpone the introduction of more abstract concepts like classes or inheritance. 
%
%we propose an enhancement to this learning path focusing on the transitions between the stages of the course. 
%We also present a new educational programming language named Wollok, which allows maximizing the accuracy in the selection of concepts to present. 
%Finally, Wollok is accompanied by a programming environment which has lots of tools to guide the student and to help detecting mistakes
%and at the same time is in line with the most common professional practices.


%Also, educative tools that differ greatly from those used in the IT industry 
%can weaken student interest,
%as well as hamper the application of the learned concepts and techniques 
%in subsequent labor experiences. 
