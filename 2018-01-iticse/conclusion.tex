\section{Conclusion}
\label{sec:conclusion}

% In this paper, we looked at problem P with this context and these
% constraints. We proposed solution S. It has such good points and such not so
% good ones. 

The Wollok language and IDE have been put into practice for already two years, targeting hundreds of students.
They have been successful in supporting an incremental learning path, 
allowing the users to train their OO modelling skills using a very simple programming model 
and providing a smooth transition to industrial languages and associated tools.

%Defining our own programming language, allows us to give full support to the selected learning path, 
%avoiding the need of explaining complex concepts too soon in the course or forcing the student to write \emph{boilerplate code} which he cannot yet understand.

The IDE allows students to program in a controlled environment which helps them to avoid getting stuck, 
frequently guiding them to use the best programming practices.
Also, it provides a controlled environment which empowers students to use their intuition, test their ideas and explore new possibilities.
The combination of REPL and tests, we have succeeded in completely avoiding the need for I/O or undesired debugging practices, such as the inclusion of \code{println} expressions along the code.

By providing simplified versions of several industry-like tools, the Wollok IDE allows to introduce professional development practices early on in the curricula,
helping the students in getting familiarized with the kind of practices and environment they find both in later subjects as in their professional jobs.
%In our experience, good students often do not automatically become efficient professionals because they encounter difficulties in translating their academical knowledge into their professional practice.
%Letting them work with industry-like tools helps them to make this transition easier.

%\section{Future Work}
% Now we could do this or that.
%\label{sec:furtherWork}
\medskip
The main focuses of attention for the Wollok IDE development team are the detection of programming errors and bad practices, and the provision of quick fixes, content assistance and refactorings.
A cornerstone to achieve these goals is the type inferer, which is one of our current main objectives.
Still, providing a type inferer for a language such as Wollok has many subtleties, which deserve an independent study \cite{passerini_nicolas_extensible_2014}.
%Also, we plan to include an \emph{effect system} \cite{nielson_type_1999}.

Other future tasks include
extending the REPL to be a full-fledged editor which re-evaluates expressions after a code change, displaying the new results (such as Scala Worksheets\footnote{\url{https://github.com/scala-ide/scala-worksheet}});
improving team work support,
and a web-based version of the interpreter.

%Another characteristic of programming in the real world is the need to work in teams. 
%The success of object-oriented languages is partly due to their advantages in group projects. 
%It is necessary to teach our students about the techniques needed for teamwork, right from the beginning. 
%To do this, it is essential that the environment has some form of support for group work \cite{kolling_problem_1999}.
%Therefore, we plan to create simplified tools to integrate Wollok with \emph{version control systems}.

%Also there are some initiatives to build web-based or lighter versions of the IDE and the interpreter.
%This will allow Wollok to be used in context where the availability of powerful computers is restricted.
%There exists a (limited) Wollok web editor integrated to the Mumuki\footnote{\url{http://mumuki.io/}} platform, which has not been yet fully tested with students.
