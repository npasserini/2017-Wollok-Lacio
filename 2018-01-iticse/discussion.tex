%\section{Discussion}
\section{To IDE or not to IDE}
\label{sec:discussion}

% Discussion of actual solution \emph{vs.} initial constraints from \ref{sec:problem}. Explain the space of the solution, why we made it this way.
% Evaluation of the solution. How does the solution meet the criteria? Where does it succeed or fails...

%\subsection{A brand new language}
%\label{sec:newLanguage}
%% Hacer o no hacer un lenguaje nuevo.
%A common point of controversy is whether it is worth to create a brand new language and toolset
%instead of building our pedagogical ideas on top of existing ones, such as Self, Ruby, Smalltalk or even Eiffel.
%In our experience, beginner programmers require different features from their working environment that advanced ones.
%The right selection of tools and concepts can produce substantial improvements in the learning process.
%Therefore, we believe that the possibility of fine tuning provided by a specialized environment largely pays for the additional effort.
%
%These considerations made us favor the crafting of an ad-hoc language over choosing existing ones, even those allowing to define WKOs like Self or Ruby.
%
%Each semester, a group of more than 20 teachers in 3 different universities share their experience with the language and tools and discuss about new features and changes to the system. 
%Every modification is guided by a shared understanding about how to teach OOP \cite{lombardi_instances_2007,lombardi_carlos_alumnos_2008,griggio_programming_2011,spigariol_lucas_ensenando_2013}.
%% y que muestran las grandes posibilidades que se dan a partir de esta decisión inicial: imports, tests y manejo de propiedades.
%
%A good example of teaching-specific language-design decisions is Wollok import system,
%\ie the way that a programming language allows the programmer to refer in one unit of code (for example a file) to program entities defined elsewhere.
%The import system allows the student to write his first very simple programs without knowing about packages or modularization, which are far too complex for him at the beginning. Still, later in the course modularization concepts are introduced and even the language forces the student to separate his code in different units. 
%A full description of how the import system works and other syntax decisions can be found in \cite{javier_fernandes_wollok_2014}.

%\subsection{To IDE or not to IDE}
A frequent controversy between software programmers is about the convenience of using an IDE or a simpler text editor for writing code. In the last decade, several languages, frameworks and other tools have become popular for which there are fewer visual or integrated environments. 

This scarcity of tools has diverse roots. In some cases, the lack of type information undermines the possibility to implement features such as code completion, automatic refactorings or code navigation.
In other cases, languages and frameworks change so fast that tools can not catch up.
Frequently, there is also a matter of taste.
Also, some teachers argue that providing the student with too many tools will make him dependent on those tools.

In our view, tools that simplify day to day work can not be neglected. 
We found that professional programmers use many tools that help them 
program consistently and efficiently.
Many popular text editors allow for additions in the form of \emph{plugins}, 
where the programmer can create his own personalized development environment.
Other tools that are not integrated into the development environment, 
are inserted into the development process by other means; 
\eg a continuous integration process may run a \emph{linter} on each commit, check the build and run tests.
So, instead of a discussion about whether we need powerful tools, we rather see an evolution from heavy monolithic environments
onto an ecosystem of light tools that allow the developer or team to create a unique environment 
which accomodates to their specific needs and taste. 

Still, in our specific case, we opted for an \emph{integrated} environment because it simplifies the set up for beginners. 
In more advanced courses, we think that it could be a good idea to let the students build their own environments.

%\subsection{Enhancing programming practices}
%In our opinion, good programming practices should be taught from the very beginning of programming curricula.
%In fact, we claim that the teaching of programming concepts, principles and techniques should be \emph{integrated} with that of software development practices and habits, forming a single body of knowledge.
%%This claim is independent from the recurrent debate about rules and conventions. 
%Our experience shows that it is unlikely that novice programmers appreciate the advantages of, 
%\eg good variable names or correct code indentation, as these attributes are more easily appreciated on larger programs.
%We remark that good practices include the right selection of development tools and the proper use of them. 
%This consideration greatly influenced the decision of offering a complete IDE as part of Wollok.
%
%Some of the features of the Wollok language and IDE have been conceived with the objective of promoting good coding habits. 
%We mention the proposal of a proper indentation scheme, 
%the signaling of dead code and other possible bad smells as warnings in the code editor, 
%and the imposition of some degree of organization in class definitions, described in Section~\ref{sec:environment}.
%In turn, the inclusion of a simple modularization scheme in the language and the set of code navigation and visualization facilities included in the IDE 
%promote the care for good code organization in a software project.

%Finally, we think that teaching programming should include teaching the best practices that we see in the professional world. A student which knows the best practices and tools that are used in professional software development will have a significant advantage over those who lack these knowledge.

%\subsection{Image vs. files}
%Unlike many traditional OO programming environments, which are image-based, Wollok is file-based.
%While we have found solid grounds for taking this decission (\cf \secref{file-based}), 
%we also recognize the importance of a \emph{live object environment}, 
%\ie a work space in which the programmer can interact with live objects by sending them messages.
%As in many file-based OO and scripting languages, in Wollok this kind of interaction is achieved through an interactive console or 
%REPL\footnote{Read-eval-print-loop \cite{hey2014computing}.}
%The interactive console allows the programmer to inspect the state of his program or modify it, both at the end of an execution or in the middle of a debug session.
%However, right now we do not provide a way to modify the program while it is running, as it happens in classigal image-based environments.
%This kind of features have been postponed because we we think that modifying the code in the middle of a program run 
%usually has subtle consecuences that are difficult to grasp for an unexperienced programmer.
%It is not uncommon to see that students get confused when they try to modify live code, 
%so there is a high price to pay with little to gain in return.
