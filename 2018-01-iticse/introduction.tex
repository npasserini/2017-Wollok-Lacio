\section{Introduction}
\label{sec:intro}

% Context
% (a) hay que arrancar introduciendo nuestra visión, contando por qué enseñamos objetos de determinada manera, por qué hicimos Ozono/Wollok.

% La importancia de arrancar con objetos.
%\emph{Object-oriented programming} (OOP) has become the \textit{de facto} standard programming paradigm in industrial software development.
%Therefore, in the last years software engineering curricula have put more emphasis in object-oriented courses.

% Problem
%- aprender a programar se ha revelado como una tarea difícil.
Teaching how to program has revealed itself a difficult task~\cite{dijkstra_89a, jenkins2002difficulty}.
We have individualized three specific aspects present in many initial programming courses that hinder  the learning process: 
a complex programming language,
too many concepts needed for a first working program and
programming environment that are not conceived for the specific needs of an initial student~\cite{singh2012}.

%First, OO courses tend to focus too much on syntax and the particular characteristics of a language, instead of focusing on OOP distinctive characteristics.
%Second, many OO languages used in introductory courses do require to grasp a lot of quite abstract concepts before being able to build a first program.
%Finally, poor programming environments are used, although we are at a time where an unexperienced programmer could be making great use of the guidance a good programming environment could provide. These problems are not exclusive of OO courses, they are present in all the general programming courses.

%- múltiples iniciativas de entornos y lenguajes pedagógicos, para distintos públicos y con distintos objetivos.

\medskip 

%\section{Related Work}
%\label{sec:related}

% Other solutions in the domain, and a real comparison of our contribution with solutions from other people.
BlueJ \cite{bennedsen_bluej_2010} is an educative environment for programming in Java 
which shares several points of view with our approach (\cf \secref{related}).




Other approaches have put their focus in the 
A step further is to provide a whole programming environment specifically designed to aid novice programmers.

 
such as Squeak \cite{ingalls_back_1997}, 
Loop \cite{griggio_programming_2011}, 
and BlueJ .

Other environments make use of block-based or visual programming, 
such as Scratch \cite{malan_scratch_2007}, Etoys \cite{lee_empowering_2011} and Kodu \cite{kodu}. 
In our vision, these tools are suitable for stimulating interest in programming and for being used in secondary education, but not beyond that stage.

Other educators propose to use industrial languages in introductory courses, 
such as Java \cite{kolling2001guidelines}, Eiffel \cite{meyer1993towards}, Smalltalk \cite{ducasse2006squeak} and Self \cite{Unga87a}.


\section{Related Works}
\label{sec:related}

% Primero hablar de lenguajes específicos para enseñar
There have been proposals to tackle the first two problems by defining specific languages
that provide simplified programming models.
such as Karel++~\cite{bergin_karel++:_1996} and Mama~\cite{harrisonmama}.
This approach has been used even outside the OO world \cite{feurzeig_programming-languages_1970, pattis_karel_1981, lopez_nombre_2012}.
In this paper we use based on \emph{Wollok} \cite{passerini2017wollok}, 
an educative OO language designed to support a novel path to introduce OO concepts \cite{lombardi_instances_2007,lombardi_carlos_alumnos_2008,spigariol_lucas_ensenando_2013}.
This alternative learning path proposes to focus first on objects, messages an polymorphism, 
while delaying the introduction of more abstract concepts, such as classes, types or inheritance.

This approach has been used even outside the OO world \cite{feurzeig_programming-languages_1970, pattis_karel_1981, lopez_nombre_2012}.

\cite{passerini2017wollok}, 
which consists on a novel path to introduce OO concepts focusing first on objects, messages and polymorphism 

that supports an iterative learning path allows for a extremely simple initial programming model, 
as well as smooth transitions to a complete OO dynamicaly typed language.

which consists on a novel path to introduce OO concepts focusing first on objects, messages and polymorphism 

that supports an iterative learning path allows for a extremely simple initial programming model, 
as well as smooth transitions to a complete OO dynamicaly typed language.

% Environments
Still, the language by itself can not
A step further is to provide a whole programming environment specifically designed to aid novice programmers 
such as Squeak \cite{ingalls_back_1997}, 
Traffic \cite{broy_outside-method_2003},
Loop \cite{griggio_programming_2011}, 
and BlueJ \cite{bennedsen_bluej_2010}.

Other environments make use of block-based or visual programming, 
such as Scratch \cite{malan_scratch_2007}, Etoys \cite{lee_empowering_2011} and Kodu \cite{kodu}. 
In our vision, these tools are suitable for stimulating interest in programming and for being used in secondary education, but not beyond that stage.

Other educators propose to use industrial languages in introductory courses, 
such as Java \cite{kolling2001guidelines}, Eiffel \cite{meyer1993towards}, Smalltalk \cite{ducasse2006squeak} and Self \cite{Unga87a}.
%Self, at the same time, has pioneered in allowing for OOP without classes.


%There have been several proposals to address the difficulties in introductory OO courses 
%by defining a specific language which provides a simplified programming model such as Karel++~\cite{bergin_karel++:_1996} and Mama~\cite{harrisonmama}.
%This approach has been used even outside the OO world \cite{feurzeig_programming-languages_1970, pattis_karel_1981, lopez_nombre_2012}.
%% Environments
%A step further is to provide a whole programming environment specifically designed to aid novice programmers 
%such as Squeak \cite{ingalls_back_1997}, Traffic \cite{broy_outside-method_2003} and BlueJ \cite{bennedsen_bluej_2010}. 


\medskip

% Nuestro trabajo
Our pedagogical approach follows the work of Lombardi \etal
\cite{lombardi_instances_2007,lombardi_carlos_alumnos_2008,griggio_programming_2011,spigariol_lucas_ensenando_2013,passerini2017wollok}, 
which consists on a novel path to introduce OO concepts focusing first on objects, messages and polymorphism 
while delaying the introduction of more abstract concepts,
such as classes, types or inheritance.
This way of organizing a course provides a more gentle learning curve to students and allows them to write completely working programs from the first classes.
Also, this approach gives great importance to the programming tools used in the course, 
stating that they should be carefully selected and customized, 
taking into account the specific needs of beginner programmers,
as well as the intended pedagogical view.

As a result, we conceived both a programming language and an accompanying integrated development environment (IDE) that closely follow the aforementioned pedagogical approach. 
Also, this new toolset is designed to overcome some disadvantages 
found in previous projects with similar approaches,
most noticeably related to two gaps not adecquately covered by the available tools:
(a) between the initial, simplified programming model and the classical OOP model
used in the late part of the courses, and
(b) between the experience in the classroom and the reality in (most) professional environments.
We consider that a essential part of this work is resolving the apparent contradition
between customizing language and environment for student needs
and at the same time keeping them close enough to their professional counterparts.

% What our solution is, so that the reader knows where the paper is going.
% Contribution of the paper

The main goal of this paper is to describe how the set of tools included in the Wollok IDE%
\footnote{
	\url{http://www.wollok.org/}. 
	Source code and documentation can be found in Github 
	(\url{https://github.com/uqbar-project/wollok}).
	Wollok is open-source and distributed under LGPLv3 License 
	(\url{http://www.gnu.org/copyleft/lgpl.html}).}, 
contributes to support a gentle and industry-aware introduction to OOP.

%Wollok combines some characteristics that are typically present in academical environments with others that are more easily seen in industrial ones, aiming to enrich them to fulfill the specific needs of novice programmers and the proper development of a first OOP course.
%Its most noticeable characteristics are:
%(1) the possibility of using self-defined objects to support an introduction to the topic that postpones the introduction of the concept of class,
%(2) the possibility of combining self-defined with class-based code objects in the same program,
%(3) the decission of offering an IDE with edition and code management capabilities that are fine-tuned for unexperienced programmers
%and (4) that both the language as the IDE share similarities with their industrial counterparts, in order to soften the later transition to the professional tools.

\medskip 
% Paper structure
%The rest of the paper is structured as follows. 
In \secref{problem} we present the problems of learning Object Oriented programming, and the consequences of this difficulties to the students. \secref{wollokLanguage} describes the proposed language and design goals. 
In \secref{environment} we describe the integrated development environment we have developed for Wollok and all the features it has and how they are useful for the teaching of programming skills. \secref{discussion} analyses the different design decisions we have taken.
%, while \secref{related} compares our solution with another similar approaches. 
Finally, we summarize our contributions in \secref{conclusion},
along with some possible lines of further work derived from this initial ideas. 
%As an appendix we have Sections \ref{sec:implementation}, \ref{sec:wollokSoftware} and \ref{sec:ChecksAndValidations} which give more details about the implementation.


