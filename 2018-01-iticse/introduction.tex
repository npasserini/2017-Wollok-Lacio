\section{Introduction}
\label{sec:intro}

% Context
% (a) hay que arrancar introduciendo nuestra visión, contando por qué enseñamos objetos de determinada manera, por qué hicimos Ozono/Wollok.

% La importancia de arrancar con objetos.
%\emph{Object-oriented programming} (OOP) has become the \textit{de facto} standard programming paradigm in industrial software development.
%Therefore, in the last years software engineering curricula have put more emphasis in object-oriented courses.

% Problem: aprender a programar es difícil
Teaching how to program has revealed itself a difficult task~\cite{dijkstra_89a, jenkins2002difficulty}.
We have individualized three specific aspects present in many initial programming courses that hinder the learning process: 
(a) a complex programming language,
(b) too many concepts needed for a first working program and
(c) programming environment that are not conceived for the specific needs of an initial student~\cite{singh2012}.
While some of these problems are general to any initial programming course, 
our main focus are object-oriented programming (OOP) courses.

%First, OO courses tend to focus too much on syntax and the particular characteristics of a language, instead of focusing on OOP distinctive characteristics.
%Second, many OO languages used in introductory courses do require to grasp a lot of quite abstract concepts before being able to build a first program.
%Finally, poor programming environments are used, although we are at a time where an unexperienced programmer could be making great use of the guidance a good programming environment could provide. These problems are not exclusive of OO courses, they are present in all the general programming courses.

\smallskip

% Known tracks for solutions
% here you want to show that you are not an idiot not knowing what have been around
There have been proposals to tackle the first two problems by defining specific languages
that provide simplified programming models \cite{bergin_karel++:_1996,harrisonmama}.
Our work is based on \emph{Wollok} \cite{passerini2017wollok}, 
an educative language designed to support a novel path to introduce OO concepts \cite{lombardi_instances_2007,lombardi_carlos_alumnos_2008,spigariol_lucas_ensenando_2013}
This alternative learning path proposes to focus first on objects, messages an polymorphism, 
delaying the introduction of more abstract concepts, such as classes, types or inheritance
(\cf \secref{problem}).

% Environments
Also, there are pedagogical approaches that focus on the programming environments.
Some of them, propose to use industrial environments 
\cite{broy_outside-method_2003,ducasse2006squeak,Unga87a,ingalls_back_1997},
but beginners have specific needs 
that frequently are not adecquately addessed by industrial tools.
On the other hand, there have been proposals of educative environments
\cite{griggio_programming_2011,malan_scratch_2007,bennedsen_bluej_2010}
that cover student needs but differ too much from their industrial counterparts,
frequently making the transition difficult for students.

\smallskip

% Factual solution tracks, to position...
The main goal of this paper is to describe the \emph{Wollok IDE}%
\footnote{
	\url{http://www.wollok.org/}. 
	Source code and documentation can be found in Github 
	(\url{https://github.com/uqbar-project/wollok}).
	Wollok is open-source and distributed under LGPLv3 License 
	(\url{http://www.gnu.org/copyleft/lgpl.html}).},
a programming environment that supports a gentle introduction to OOP,
as well as to escort the student in the path to more complex, industry-like 
programming models and tools.

By using an educative language frees the IDE from coping 
with advanced features (\eg \emph{reflection)} that are common in industrial OO languages, 
but not required in an introductory course.
This, in turn, allows for more customized tools, 
\eg which can provide better feedback for some typical errors (\cf \secref{detectMistakes}).
Finally, the \emph{Wollok IDE} integrates software engineering tools,
such as unit testing or code versioning and sharing,
that allow for a first introduction to these thechniques.

% What our solution is, so that the reader knows where the paper is going.
% Contribution of the paper

%Wollok combines some characteristics that are typically present in academical environments with others that are more easily seen in industrial ones, aiming to enrich them to fulfill the specific needs of novice programmers and the proper development of a first OOP course.
%Its most noticeable characteristics are:
%(1) the possibility of using self-defined objects to support an introduction to the topic that postpones the introduction of the concept of class,
%(2) the possibility of combining self-defined with class-based code objects in the same program,
%(3) the decission of offering an IDE with edition and code management capabilities that are fine-tuned for unexperienced programmers
%and (4) that both the language as the IDE share similarities with their industrial counterparts, in order to soften the later transition to the professional tools.

\smallskip
% Paper structure
%The rest of the paper is structured as follows. 
In \secref{problem} we present the problems of learning Object Oriented programming
together with a brief introduction to our pedagogical approach. 
\secref{methodology} describes the iterative methodology 
used for the development of our pedagogical approach 
and the tools that support it,
while \secref{environment} describes some characteristics of these tools
and their role for teaching programming skills.
\secref{related} compares our solution with another similar approaches, while
\secref{results} describes the results obtained.
\secref{discussion} analyses some key design decisions
and \secref{conclusion} summarises our contributions
along with some possible lines of further work.
%As an appendix we have Sections \ref{sec:implementation}, \ref{sec:wollokSoftware} and \ref{sec:ChecksAndValidations} which give more details about the implementation.


