\section{Methodology}
\label{sec:methodology}

Nuestra metodología de validación se compone de dos partes. 
En una primera etapa realizamos un análisis cualitativo, basado en entrevistas a estudiantes y docentes, con un conjunto de preguntas semiabiertas. 

Entre los docentes entrevistaron se incluyeron tanto docentes que habían utilizado la metodología propuesta como otros que no. En todos los casos, los docentes tienen alguna experiencia con la metoodología tienen experiencia previa con otros métodos, lo que nos permite también realizar preguntas que comparen los resultados en ambos casos.
% Esto no sé si hace falta.
Cabe aclarar que a pesar de que existe suficiente documentación sobre la propuesta metodológica, cada docente realiza una implementación propia, por lo que no todos los cursos se llevan acabo exactamente de la misma manera.

En el caso de los estudiantes, se entrevistó a estudiantes que estaban a punto de terminar un curso que seguía la metodología propuesta y se los consultó sobre la experiencia en este curso, comparada con otras materias, las dificultades que encontraron en cada caso y las posibilidades de aplicar estos conocimientos que han tenido en otros ámbitos. También nos interesa medir el resultado final; en este sentido nos interesa una visión teórico-práctica, definida por la capacidad de la \emph{aplicación consciente} del contenido teórico, es decir, debe ser capaz de resolver problemas concretos pudiendo defende                                          r desde su comprensión de la teoría la solución que propone.
Las entrevistas fueron realizadas en todos los casos por docentes que no participaban de cada curso en cuestión y garantizando la anonimicidad de las respuestas, para permitir que tuvieran total libertad para expresar su visión. Asimismo se buscó formar una selección de estudiantes que fuera representativa de cada curso en cuanto a las calificaciones finales obtenidas.

\medskip
El objetivo de la segunda etapa es realizar un análisis cuantitativo que permita validar o refutar las primeras conclusiones del análisis cualitativo.

A partir del análisis de las respuestas obtenidas en la etapa anterior, se formuló un cuestionario para estudiantes, que luego fue distribuidos en una decena de cursos en 3 universidades nacionales, obteniendo unas 140 respuestas.

Adicionalmente, se recolectaron datos de retención y aprobación de diferentes cursos de programación.


