\section{Methodology}
\label{sec:methodology}
Wollok language and IDE are developed in an iterative process, 
guided by our pedagogical approach 
and at the same time providing the basis for a classroom experience, 
which in turn provides feedback to the process.

After each university semester, both students and teachers are surveyed.
Questionnaires are designed to determine which are the topics that result more difficult for students, 
and to analyse the relationship to other variables,
such as the order in which concepts are presented,
the practice and examples used in each stage of the course
and the support tools that could help grasping each concept.

Then, the results of surveys is analyzed by a board of teachers from five universities.
Although each teacher has his own way in front of the course, there is an extensive basis of agreement.
This consensus allows us to create shared teaching supplies, such as theoretical material, sample exercises and exams and other support tools.
In the last years, a big amount of this effort has been specifically devoted to define and build the Wollok language and IDE.

\medskip

During 2017, 140 students answered the surveys, 
including also some courses that do not use Wollok.
Most of them were surveyed twice: 
before and after the course, 
in order to follow the evolution of each student
and to analyse the relationship between their previous experience 
and their perception of the tools.
Surveys are anonymous to ensure that students are not afraid of posting negative critics%
\footnote{Students that answered both initial and final course surveys were asked to identify themselves using a nickname, allowing us to correlate their responses.}.

%En el caso de los estudiantes, se entrevistó a estudiantes que estaban a punto de terminar un curso que seguía la metodología propuesta y se los consultó sobre la experiencia en este curso, comparada con otras materias, las dificultades que encontraron en cada caso y las posibilidades de aplicar estos conocimientos que han tenido en otros ámbitos. 
%También nos interesa medir el resultado final; en este sentido nos interesa una visión teórico-práctica, definida por la capacidad de la \emph{aplicación consciente} del contenido teórico, es decir, debe ser capaz de resolver problemas concretos pudiendo defender desde su comprensión de la teoría la solución que propone.

The information gathered from surveys has been complemented by individual interviews.
To anonymity, interviews are always conducted by a teacher of another university.
Interviews were guided by the same questions as in the survey, 
but allowing for longer, open responses 
and giving the opportunity for the interviewer to deepen some topics, 
depending of the student answers and background.

% Asimismo se buscó formar una selección de estudiantes que fuera representativa de cada curso en cuanto a las calificaciones finales obtenidas.

% Ideas para el futuro
% - Entrevistas a docentes
% - Estudiantes del curso siguiente.

Adding up these techniques allows us to gather different flavors of data, 
combining the statistically more significant data obtained from surveys and 
with the deeper insights that can be obtained in an interview.
Also, interviews allows to discover oportunities for improving the questions in future surveys.

Finally, we collected retention and approval data, along with final exams,
from courses using different approaches, languages and other tools.
The results of this process are detailed in Sec. \ref{sec:results}.
