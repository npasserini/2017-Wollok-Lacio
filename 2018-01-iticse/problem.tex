%\section{Problem Description}
\section{Why Wollok?}
\label{sec:problem}

% Context, exposed with the \textbf{most precise terms possible} (don't open unwanted doors for the reader)
% Probably set the vocabulary before to cut any misinterpretation

%En la misma dirección, propongo una sección 2 bien cortita, donde nos limitemos a señalar los aspectos que apoyan lo que vamos a decir de Wollok. Y tal vez no todos, contar solamente 
% Los cursos se enfocan en sintaxis y usan lenguajes inadecuados.
One cause behind the difficulties in learning OOP 
is the use of languages that require the student to understand a lot of abstract concepts 
before being able to produce the simplest program \cite{kolling_problem_1999}.
% Ejemplo con Java.
Figure \ref{fig:helloWorld} shows an example of a possible first program, written in Java \cite{arnold_java_1996}.
Before being able to have a first object and send a message to it,
the student has to deal with packages, classes, variable scopes, types, arrays, 
printing to standard output and class methods. 
Courses tend to spend too much time on the details of a specific language, 
leaving too little time to become fluent on the distinctive featuresof OOP,
such as identifying objects and their knowledge \emph{relationships}, 
assigning \emph{responsibilities} 
and the power of \emph{encapsulation} and \emph{polymorphism} 
to build robust and extensible software.

% Además necesitamos environments
Moreover, frequently the students do not have proper tools that could help them to overcome all the obstacles.
This has even more importance 
if we want courses to deal with larger programs 
and to teach concepts such as testing, debugging and code reuse~\cite{kolling_problem_1999}.
 
For example, writing to standard output would be considered a problem in industrial software construction and teaching the students to try out their programs in this way 
introduces a bad practice,
which will have to be \emph{unlearned} later.
Still, some kind of user interaction is required 
in order to see the behaviour of the program,
even if proper handling of it is beyond the scope of an initial OO course.


\begin{figure}[ht]
\vspace{-3mm}
 \centering
 \begin{lstlisting}[language=Java]
	package examples;
	
	public class Accumulator {
		private int total = 0;
		
		public int getCurrentTotal() { return total; }
		public void add(amount) { total += amount; }

		public static void main(String[] args) {
			Accumulator accum = new Accumulator();
			accum.add(2);
			accum.add(5);
			accum.add(8);
			System.out.println(accum.getCurrentTotal());
		}
	}\end{lstlisting}
\vspace{-3mm}
\caption{Sample initial Java program which diverts student attention from the most important concepts.}
\label{fig:helloWorld}
\vspace{-3mm}
\end{figure}

%- en particular, el ruido que le hace a los alumnos arrancar con clases.
% TODO, no estoy seguro de cómo encararlo ni de si es lo más importante.

% Por eso los pibes no aprenden
%- la tensión entre que te vaya bien en la materia y el uso de las ideas en la industria.

% Free form, variable number of sections, technical details.
% But in general do not mix solution and discussions/possible variation let that for discussion

Wollok provides an \emph{simplified programming model} (SPM)
that allows students to create programs containing objects, 
messages and polymorphism without the need for more abstract concepts 
such as inheritance, type annotations or even classes.
Wollok also allows the incremental introduction of more abstract concepts,
providing a smooth transition into a full-fledged OO programming model.

The example in Figure \ref{fig:helloWorld/wollok} 
shows a possible first program in a Wollok-based OO introductory course.
Syntax has been reduced to a minimum and the basic constructs of the language match exactly the concepts we want to transmit, \eg \code{var} is used to define variables and \code{method} is used to define methods.
The \code{accumulator} object is defined 
as a stand-alone (\ie it has no visible class)
and \emph{well-known} (\ie globally accesible) object (WKO).
%This is also consistent with some modern industrial OO languages that allow to define both classes or standalone objects, such as Scala \cite{Oder04a}.
%To build this first program students are not required to know about typing, scoping or packaging.
%The concepts required to understand this program are no more than program, object, message and argument passing.

\begin{figure}[ht]
\vspace{-3mm}
 \centering
 \begin{lstlisting}[language=Wollok]
	object accumulator {
		var total = 0
		var evens = 0
		
		method getCurrentTotal() { return total }
		method add(amount) { 
			total += amount 
			if (amount % 2 == 0) { evens += 1 }
		}
		method evenCount() { return evens }
	}\end{lstlisting}
\vspace{-3mm}
\caption{\small Sample initial Wollok object definition.}
\label{fig:helloWorld/wollok}
\vspace{-3mm}
\end{figure}

To allow the student to interact with objects while avoiding I/O a \emph{read-eval-print-loop}~(REPL) is provided.
Shortly after in the course, we introduce unit testing, 
which slowly replaces the REPL as the main form of interacting with objects. 
These tools are described in detail in Sec. \ref{sec:testing}.

% Misceláneos
% Profundizar y pulir el highlighting the conceptos primarios y la estratificacion de conceptos
The presence of literals for lists (\eg \code{[1,2,3]}) and sets (\eg \code{\#\{1,2,3\}})
allows us to use collections and, therefore, 
increase the complexity of examples that we can build before introducing classes.
We also introduce \emph{closures} at this stage.

Afterwards, we introduce classes.
Wollok helps us in the transition: any pre-existent stand-alone object can be converted into a class by just changing the keyword \code{object} for \code{class}%
\footnote{As a matter of fact, we will also change the name, as our code convention mandates lowercase names for objects and uppercase names for classes}.
Moreover, stand-alone objects can be used in the same program and even be polymorphic with class-based objects.
Examples of all these language features can be found in \cite{passerini2017wollok}.

\medskip

% Contribution
While neither the language itself nor the programming environment 
contain novel features that are unseen in industrial tools,
the assemblage of selected features, 
each one carefully selected due to its educational value,
is not found in other previous programming environments, 
neither educational nor industrial.
Often, the rich set of tools an industrial IDE offers 
cannot be exploited by an inexperienced programmer.
Even worst: they can cause confusion.
Therefore, there is much to gain from a language and IDE that provide the exact tools 
a student can understand and take advantage of at each stage of his learning process.

At the same time, we have noticed that sometimes students
have a hard time translating their knowledge to their professional activity.
We think that a good mitigation tool ist to bring the activities in the course as close as possible to the professional practice \cite{McDermott2017AssessmentAuthenticity}.
We put special emphasis in solving the apparent contradition
between customizing language and environment for student needs
and at the same time keeping them close enough to their professional counterparts.
At the same time, we incorporate industrial best practices 
such as code repositories and unit tests, 
adapting them to the possibilities of students with little or no programming experience.
%This new toolset is designed to overcome some disadvantages 
%of previous projects with similar approaches, 
%frequently related to two gaps:
%(a) between the initial simplified programming model 
%and the classical OOP model used later in the course, and
%(b) between the experience in the classroom and the reality in (most) professional environments.

\medskip

% Constraints that influenced the solution (because the solution is not
% universal) \emph{e.g.} our requirements for a solution, possibly not all
% satisfied. They should be sound and believable. Analysis of the criteria.
% Imagine that you are another guy having this problem do the constraint
% matches yours so that you could apply the solution

The current study and development have been focused on university students which have had a previous subject on imperative programming.
The natural extension of this work is the adaptation of these ideas to teenagers or, more generally, students without any prior programming experience.

