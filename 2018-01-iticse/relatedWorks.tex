%\section{Related Work}
%\label{sec:related}

% Known tracks for solutions
% here you want to show that you are not an idiot not knowing what have been around

%- en particular en objetos hay iniciativas, tanto de presentar un modelo inicial simplificado, como de definir lenguajes y entornos de propósito pedagógico.

% Primero hablar de lenguajes específicos para enseñar
There have been several proposals to address the difficulties in introductory OO courses 
by defining a specific language which provides a simplified programming model such as Karel++~\cite{bergin_karel++:_1996} and Mama~\cite{harrisonmama}.
This approach has been used even outside the OO world \cite{feurzeig_programming-languages_1970, pattis_karel_1981, lopez_nombre_2012}.
% Environments
A step further is to provide a whole programming environment specifically designed to aid novice programmers 
such as Squeak \cite{ingalls_back_1997}, Traffic \cite{broy_outside-method_2003} and BlueJ \cite{bennedsen_bluej_2010}. 

The great differences between these programming languages and environments show that they have to be analysed in the light of the pedagogical approaches behind them.
The tools are of little use without their respective pedagogical view.

% Other solutions in the domain, and a real comparison of our contribution with solutions from other people.

% Agregar referencia al paper de Fidel, y otros

% Base tomada del paper de ESUG 2011
In recent years, several pedagogical programming environments have arised, with diverse purposes.
Some of them make use of block-based or visual programming, 
such as Scratch \cite{malan_scratch_2007}, Etoys \cite{lee_empowering_2011} and Kodu \cite{kodu}. 
In our vision, these tools are suitable for stimulating interest in programming and for being used in secondary education, and not beyond that stage.
Other approaches, such as Gobstones \cite{lopez_nombre_2012}, focus on a first universitary programming course%
\cl{ver si el artículo sobre Gobstones tiene related work}.
Finally, there are other pedagogical programming tool proposals that share our interest in a first OOP course, such as BlueJ \cite{bennedsen_bluej_2010}, Karel++~\cite{bergin_karel++:_1996}, Mama~\cite{harrisonmama} and Loop \cite{griggio_programming_2011}.\np{alguno más?}

On the other hand, other educators propose to use industrial languages in introductory courses, such as Java \cite{kolling2001guidelines}, Eiffel \cite{meyer1993towards, broy_outside-method_2003}, Smalltalk \cite{ducasse2006squeak} and Self \cite{Unga87a}.
%Self, at the same time, has pioneered in allowing for OOP without classes.


%There have been several proposals to address the difficulties in introductory OO courses 
%by defining a specific language which provides a simplified programming model such as Karel++~\cite{bergin_karel++:_1996} and Mama~\cite{harrisonmama}.
%This approach has been used even outside the OO world \cite{feurzeig_programming-languages_1970, pattis_karel_1981, lopez_nombre_2012}.
%% Environments
%A step further is to provide a whole programming environment specifically designed to aid novice programmers 
%such as Squeak \cite{ingalls_back_1997}, Traffic \cite{broy_outside-method_2003} and BlueJ \cite{bennedsen_bluej_2010}. 
