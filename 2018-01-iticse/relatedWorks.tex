%\section{Related Work}
%\label{sec:related}

% Other solutions in the domain, and a real comparison of our contribution with solutions from other people.
BlueJ \cite{bennedsen_bluej_2010} is an educative environment for programming in Java 
which shares several points of view with our approach (\cf \secref{related}).




Other approaches have put their focus in the 
A step further is to provide a whole programming environment specifically designed to aid novice programmers.

 
such as Squeak \cite{ingalls_back_1997}, 
Loop \cite{griggio_programming_2011}, 
and BlueJ .

Other environments make use of block-based or visual programming, 
such as Scratch \cite{malan_scratch_2007}, Etoys \cite{lee_empowering_2011} and Kodu \cite{kodu}. 
In our vision, these tools are suitable for stimulating interest in programming and for being used in secondary education, but not beyond that stage.

Other educators propose to use industrial languages in introductory courses, 
such as Java \cite{kolling2001guidelines}, Eiffel \cite{meyer1993towards}, Smalltalk \cite{ducasse2006squeak} and Self \cite{Unga87a}.


\section{Related Works}
\label{sec:related}

% Primero hablar de lenguajes específicos para enseñar
There have been proposals to tackle the first two problems by defining specific languages
that provide simplified programming models.
such as Karel++~\cite{bergin_karel++:_1996} and Mama~\cite{harrisonmama}.
This approach has been used even outside the OO world \cite{feurzeig_programming-languages_1970, pattis_karel_1981, lopez_nombre_2012}.
In this paper we use based on \emph{Wollok} \cite{passerini2017wollok}, 
an educative OO language designed to support a novel path to introduce OO concepts \cite{lombardi_instances_2007,lombardi_carlos_alumnos_2008,spigariol_lucas_ensenando_2013}.
This alternative learning path proposes to focus first on objects, messages an polymorphism, 
while delaying the introduction of more abstract concepts, such as classes, types or inheritance.

This approach has been used even outside the OO world \cite{feurzeig_programming-languages_1970, pattis_karel_1981, lopez_nombre_2012}.

\cite{passerini2017wollok}, 
which consists on a novel path to introduce OO concepts focusing first on objects, messages and polymorphism 

that supports an iterative learning path allows for a extremely simple initial programming model, 
as well as smooth transitions to a complete OO dynamicaly typed language.

which consists on a novel path to introduce OO concepts focusing first on objects, messages and polymorphism 

that supports an iterative learning path allows for a extremely simple initial programming model, 
as well as smooth transitions to a complete OO dynamicaly typed language.

% Environments
Still, the language by itself can not
A step further is to provide a whole programming environment specifically designed to aid novice programmers 
such as Squeak \cite{ingalls_back_1997}, 
Traffic \cite{broy_outside-method_2003},
Loop \cite{griggio_programming_2011}, 
and BlueJ \cite{bennedsen_bluej_2010}.

Other environments make use of block-based or visual programming, 
such as Scratch \cite{malan_scratch_2007}, Etoys \cite{lee_empowering_2011} and Kodu \cite{kodu}. 
In our vision, these tools are suitable for stimulating interest in programming and for being used in secondary education, but not beyond that stage.

Other educators propose to use industrial languages in introductory courses, 
such as Java \cite{kolling2001guidelines}, Eiffel \cite{meyer1993towards}, Smalltalk \cite{ducasse2006squeak} and Self \cite{Unga87a}.
%Self, at the same time, has pioneered in allowing for OOP without classes.


%There have been several proposals to address the difficulties in introductory OO courses 
%by defining a specific language which provides a simplified programming model such as Karel++~\cite{bergin_karel++:_1996} and Mama~\cite{harrisonmama}.
%This approach has been used even outside the OO world \cite{feurzeig_programming-languages_1970, pattis_karel_1981, lopez_nombre_2012}.
%% Environments
%A step further is to provide a whole programming environment specifically designed to aid novice programmers 
%such as Squeak \cite{ingalls_back_1997}, Traffic \cite{broy_outside-method_2003} and BlueJ \cite{bennedsen_bluej_2010}. 
