\section{Related Works}
\label{sec:related}

% Other solutions in the domain, and a real comparison of our contribution with solutions from other people.

% Primero hablar de lenguajes específicos para enseñar
Several approaches have recognized the value of tools to support introductory programming courses,
dating back to the creation of Smalltalk \cite{Gold84a} and, later, Self \cite{Unga87a}.
These environments are undoubtedly powerful 
and many of the tools and ideas they introduced are still used today  
\cite{ingalls_back_1997,ducasse2006squeak},
but they are too different from the main tools used today,
which hampers one of our main goals.
%Subjectively, students are less motivated to learn tools that they perceive as outmoded;
%objectively, both students and teachers have sometimes a hard time getting used to these tools that require them to think in a different way as the rest of the tools they use.
One of outstanding caracteristics of these environments to understand programs as a network of \emph{live objects}.
Live environment allow the programmer to manipulate objects or even modify their code 
while the program is running. 
This is a very powerful tools for a professional, 
but begginners can not take advantage of it 
because modifying code while its running requires 
a deep understanding of the nature of the code being changed.

Moreover, a view in which the program is represented by \emph{code} 
makes it easier for begginners to reason about their programs.
Other environments with similar views as ours, such as Loop \cite{griggio_programming_2011}, 
have difficulties to model unit tests, 
because of the interference between the live objects in a \emph{live image}
ant the requirement of unit tests to well-known initial state.
Therefore, Wollok IDE is designed to bring as much as possible features from these tools into a 
file based environment.

% Esto habla de lenguajes
%There have been proposals to tackle the first two problems by defining specific languages
%that provide simplified programming models.
%such as Karel++~\cite{bergin_karel++:_1996} and Mama~\cite{harrisonmama}.
%This approach has been used even outside the OO world \cite{feurzeig_programming-languages_1970, pattis_karel_1981, lopez_nombre_2012}.

% diagnóstico similar
Also, Traffic \cite{broy_outside-method_2003} and BlueJ \cite{bennedsen_bluej_2010} 
are educative environments that share several points of view with Wollok IDE.
Both of them propose alternate paths for introduce the main concepts;
BlueJ proposes an \emph{objects first} approach
and Traffic proposes an \emph{inverted curriculum} with an \emph{outside-in} strategy.
Both recommend to teach about design and architecture 
and to embed programming tasks in a broader \emph{software development} view, 
right from the first introductory course.

Still, both of them use industrial languages (Java and Eiffel, respectively) 
that require the students to handle several abstract concepts 
that are too complex for a begginner.
In the case of Traffic, both the language as the IDE are meant for professionals
and, in our view, too harsh for begginners.
BlueJ proposes proposes a \emph{Graphical User Interface} (GUI) to create objects.
Similar approaches are followed by 
In our experience, this strategy can lead to difficulties 
in the posterior transition to full fledged IDEs.

Other environments make use of block-based or visual programming, 
such as Scratch \cite{malan_scratch_2007}, Etoys \cite{lee_empowering_2011} and Kodu \cite{kodu}. 
In our vision, these tools are suitable for stimulating interest in programming and for being used in secondary education, but not beyond that stage.
