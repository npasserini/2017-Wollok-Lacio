\section{Results}
\label{sec:results}
Since March 2015 Wollok is being used to teach introductory OOP courses at university level.
Until 2017 it has been used in more than 30 courses in five different universities, reaching almost 1000 students.
It has also been used at highschool level.

Also, many of the ideas in Wollok where present in other tools we had previously developed, going back to 2006. 
In these 12 years, our ideas, although always evolving, have been applied by more than 100 teachers and 6000 students. 
In several cases, approval rates changed from 30--40\% to 80--90\%.
%, helping educate professionals with world-class performances both in the industrial as in the academic field. \np{No sé, suena muy agrandado?}
Still, we consider that approval rates are of little value,
without confirming that the knowledge level required to pass the course 
is at least the same as before.
Indeed, the level of the courses has been consistently increased since our methodology was implemented, 
covering more topics and requiring the students to build bigger programs with higher design quality.

%Acá podés mechar también los exámenes. Creo que el 2do parcial que se toma en Quilmes es zarpado. Se puede decir que tienen que resolver un problema que involucra aprox x clases, el uso de herencia y polimorfismo, y que la mayor parte (tirar porcentajes) lo hace en 4 horas. 

We considered the difficulty in exams as a metric of the knowledge level acquired by studends.
Before implementing our methodology, the programs required to pass an OOP exam in UTN 
used to consist of only one or two classes, with no more than 6--8 methods, and only one (fairly simple) usage of polymorphism.
Current exams require the student to write a program with 10--15 classes and no less than 20 methods (excluding getters and setters).
The number or polymorphical sets of classes/objects has been increased to 2--3, 
frequently including at least one that requires to come up with a non-obvious abstraction in order to sort out the problem.

In 2017 in UNQ, 88\% of the students that started the course took the final exam,
96\% of those where able to complete the required program (and tests) in less than four hours.
The final approuval rate was 79\%.

\medskip
% Para entender mejor las reacciones al uso de Wollok, realizamos encuestas y entrevistas, organizadas de acuerdo a lo que se indica en la sección de metodología. 
For a better understanding to the effect of using Wollok, we've realized surveys and interviews, organized according to the ideas described in Section \ref{sec:methodology}.
% Las encuestas cubrieron un universo de x estudiantes de y universidades.  ... algunos números simpáticos de las encuestas ...
We surveyed 15 courses during 2017, obtaining 133 responses.
% dificultad apreciada por los alumnos
% \emph{las fáciles}: aprobación / asistencia / notas en las entregas. 
These surveys show that 84\% of students found Wollok at least significantly easier to understand than other languages (\cf Fig. \ref{fig:surveys/languageAppreciation}),
and 95\% thinks that Wollok has helped them understand their mistakes and learn from them.
At the same time 70\% of the students consider that having learnt Wollok will help them in their professional practice.

\begin{figure}[ht]
 \centering
 \footnotesize
 \begin{tabular}{|p{13em}|c|c|c|c|c|}
 	\hline
 	& Abs. & Sign. & Part. & Not & N/A \\
 	\hline
 	Easier to learn than other langs. & 33 & 74 & 16 & 4 & 4 \\
 	Helps understanding errors & 47 & 78 & 5 & 1 & 0 \\
 	Similar to prev. known langs. & 19 & 44 & 31 & 15 & 22 \\
 	Helps professional activity & 26 & 52 & 18 & 15 & 19 \\
 	\hline
 
 \end{tabular}

 \caption{\small Student evaluation of Wollok Language and IDE. 
 For each question, each student had to select between "Absolutely", "Significantly", "Partially" or "Not at all".}
\label{fig:surveys/languageAppreciation}
\end{figure}

% Applicability
The similarity with professional practice was a major design objective of Wollok, 
as we consider it to be a major motivational point.
Students tend to decline in attention 
when they fail to see the application of the concepts that are being taught.
Interviews confirm that students have no problems recognizing this applicability;
even more, in some cases they perceive more easily the applicability of the concepts taught with Wollok 
than other concepts they had learnt using \emph{industrial} languages, such as \texttt{C++}.
For example, several students with professional experience affirmed that 
learning OOP with Wollok led them to improve their professional practice.
Others indicated that having learned OOP with Wollok was of significant help 
to learn other modern OOP languages, such as \emph{Swift}.

%3. ¿Qué le parece el lenguaje?
%Destaco => "te deja abierto a la imaginación y podés hacer cosas más interesantes" (comparando con C++)
%Otro destacado => permite olvidarse de cosas que son procedurales, que no aportaban nada para aprender. 
%Otro tópico que se repite es la simplicidad (4/6 lo mencionan), hablan de no tipado, expresiones de orden superior, no hay que incluir librerías.

% Conceptos teóricos
% justificar que la visión "primero la práctica" no elimina la base teórica, al contrario.
These insights also allow us to confirm that the subject allows the students to get a good grasp of theoretical knowledge, 
that they are aware of the new theory they incorporated
and that they are capable of applying these concepts in different technologies and situations.
We have checked this assertion in the final path of initial courses as well as in following advanced OOP courses.
For example, some courses include a final lesson in which the introduce an industrial language, such as Java.
In other courses we have introduced a (extremely simple) \emph{game-building framework} known as Wollok Game,
which requires the student to use their knowledge in new (frequently more complex) ways.

%5. ¿Qué te gustó del IDE?
% Mencionan bastante variado: los que más se repiten son outline (sorpresa?) y mensajes de error descriptivos, que ayudan a debuggear (gol), 
% después aparecen testear y pushear git, diagramas, catálogo de clases
Students assign great value to the tools of the IDE.
Some of the most appreciated tools are those that help understanding and visualization, 
such as outline, diagrams and class catalog.
The other most referenced characteristic of the IDE is its error reporting mechanism;
most students mentioned it as a great help for debugging their programs.
Finally, also integrated tooling is considered valuable, 
such as testing facilities and git integration.

% - que Wollok los ayuda, que el IDE les cae bien
% - que las demandas son sofisticadas.
% O sea, armar el discurso en sintonía con lo que se viene diciendo.

%6a. Conceptos más importantes de objetos: 
%Todos dicen polimorfismo, siempre antes que herencia.
%Casi todos hablan de responsabilidades.
%Menos se mencionan encapsulamiento, delegación y declaratividad.

%6b. Conceptos más difíciles:
%Si bien no hay muchas repeticiones, la dificultad está donde tiene que estar: asignar responsabilidades, modelar relaciones, usar el polimorfismo para no repetir código.
%Refuerza (tal vez) el putno de que los problemas sintácticos o de bajo nivel no están presentes y eso nos permite pensar en temas más interesantes.

%Lo negativo
%4. Críticas al lenguaje
%
%a. Las que lo distancian de lenguajes conocidos previamente
%- no hay bucles for/while
%- crear objetos con "object"
%- alguno extrañaba el punto y coma
%
%b.¿qué le falta?
%  4- Interfaces explícitas (6 propone usar clases abstractas)
%  5- Clases públicas y privadas
%
\medskip
On the downside, 
% Hay muchas cosas que tardaron en descubrir o no encontraron.
we discovered that students sometimes have difficulties finding some of the features of the IDE.
% Ante la pregunta ¿qué le falta? hay ideas interesantes pero aisladas, sin muchas coincidencias: más documentación, 
We consider that this problem can be alleviated by both improving student documentation and building better teacher guidelines.

% También mencionaron que el IDE tiene demasiadas cosas que no usaron y problemas para importar desde git (justo una parte que no está tuneada).
Also, they criticized that the IDE can be confusing because of having too many tools they do not know how to use.
This is a known issue, due to an implementation trade-off: 
we intend to build a minimalistic IDE in which each tool is precisely selected according to student needs,
but it would require a significantly bigger amount of work than current, Eclipse-based implementation.
Even so, Eclipse is a customizable platform and the depuration of the IDE to remove superflous tools is a work in progress.


% algún mensaje no tan descriptivo, mejores relaciones en el diagrama estático, ayuda para armar los imports.
% Y pidieron más autocompletar, y más sugerencias (lo que muestra que nos falta algo de trabajo ahí, pero también que la mayoría de los pibes valoran eso).
Other critics asked for improvements in error messages, autocompletion and smart suggestions, 
which shows that the students perceive the added value of this kind of tools.

% usarlo desde el editor sin el IDE, 
% Varios detectaron problemas relacionados con la estabilidad o lentitud. 
