One very important feature a beginner requires from her programming environment is \emph{discoverability}, 
\ie the tools should help discover possible paths of action and gently provide feedback when the student makes a mistake, 
helping her to understand what was wrong and how to fix the program.



BlueJ es un IDE pedagógico, de características bastante distintas a Wollok. En parti-
cular, no es file-based, llegás al código a partir de un diagrama de clases que parece ser
su ventana inicial. Para probar, en lugar de REPL, hay un panel con representación de
objetos a los que se puede enviar mensajes “a la Naked Objects”.

En BlueJ (ver [KQPR03] se muestran los errores en la línea en que se producen. Si
entendí bien, para ver los errores hay que compilar, o al menos así era en 2003. Habría
que ver/probar/leer ahora.

Arriesgo un resumen: partimos de diagnósticos similares, nos planteamos objetivos
similares, definimos formas de lograrlos con diferencias relevantes.

no definen un lenguaje propio, usan Java sin cambios.
en particular, no tiene una sintaxis para definición de objetos.
está despegado de los IDE industriales. 

En particular, no es text-based. La ventana
principal es un diagrama de clases, y de ahí vas a editar cada clae. 

Dos de las tres debilidades que menciona el artículo son cuándo y cómo salir de BlueJ.
BlueJ tiene un panel gráfico donde se pueden tirar objetos y enviarles mensajes . a
la Naked Objects". Wollok tiene el REPL.

Diferencias del enfoque pedgógico

Ni polimorfismo ni referencias están mencionados como conceptos principales. De
hecho no están mencionadas en el artículo. La lista de "big concepts"(sección 6) es:
classes, methods, parameters, invocation.

El camino de graduación de complejidad en la propuesta BlueJ pasa por qué hace el
estudiante: primero envía mensajes, después modifica métodos, después implementa
métodos, después agrega métodos, después agrega clases, 
recién al final, en el 2do
semestre (OK que haciendo CS1 en objetos) un proyecto desde cero. 
Trabajando siempre con proyectos grandes. 

Wollok propone ir agregando elementos al marco conceptual y el lenguaje usados: 
primero un objeto solito, después objetos que interactúan y polimorfismo, 
después colecciones con foco en referencias, después clases, después herencia.

Similitudes en la concepción
la cantidad de conceptos involucrados en el "Hello World"se destaca como un pro-
blema que debe ser esquivado.

se destaca que un problema importante es la potencial confusión entre objeto y clase,
y se propone un enfoque object first". 


La forma de lograr este objetivo difiere gran-
demente: BlueJ propone interactuar con objetos por medio de una interfaz gráfica,

Wollok propone una sintaxis de objetos que tiene una evolución natural a clases.
en ambos casos se destaca la complejidad, o bien falta de support, de los entornos
de desarrollo como un problema relevante, y se propone un entorno simplificado,
adecuado a un estudiante de primer año.

se organiza un curso graduando la conplejidad que se introduce, y en espiral. El
camino que proponen es muy (pero muy) distinto.

Similitudes en el entorno
ambos presentan los errores en la línea en que se producen.
los dos incluyen soporte para unit testing.
tal vez (está como would-be en el paper de BlueJ) los dos provean soporte para repo
de código.


En la secuencia de BlueJ el diseño aparece en la última etapa. Se aclara que es en un
proyecto grupal y con fuerte asistencia