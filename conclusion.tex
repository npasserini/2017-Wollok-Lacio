\section{Conclusion}
\label{sec:conclusion}

% In this paper, we looked at problem P with this context and these
% constraints. We proposed solution S. It has such good points and such not so
% good ones. 

The Wollok language and IDE have been put into practice for already two years, targeting hundreds of students.
They have been successful in supporting an incremental learning path, allowing the users to train their OO modelling skills using a very simple programming model and providing a smooth transition to more complex models.

%Defining our own programming language, allows us to give full support to the selected learning path, 
%avoiding the need of explaining complex concepts too soon in the course or forcing the student to write \emph{boilerplate code} which he cannot yet understand.

The IDE allows students to program in a controlled environment which helps them to avoid getting stuck, shows them best programming practices and empowers them to use their intuition and test their ideas.
We have found that often students are afraid to search for solutions not seen in the class or test their own ideas, 
which leads them to restrict themselves into a smaller set of concepts and tools they feel more secure about.
A controlled environment empowers students to look around and explore new possibilities.

Finally, the IDE provides customized versions of several industry-like tools, helping the students in getting familiarized with the kind of programming environment they will find in professional jobs.
In our experience, good students often do not automatically become efficient professionals because they encounter difficulties in translating their academical knowledge into their professional practice.
Letting them work with industry-like tools helps them to make this transition easier.

%\section{Future Work}
% Now we could do this or that.
%\label{sec:furtherWork}
\medskip
The main focuses of attention for the Wollok IDE development team are the detection of programming errors and bad practices, and the provision of quick fixes, content assistance and refactorings.
A cornerstone to achieve these goals is the type inferer, which is one of our current main objectives.
Still, providing a type inferer for a language such as Wollok has many subtleties, which deserve an independent study \cite{passerini_nicolas_extensible_2014}.
Also, we plan to include an \emph{effect system} \cite{nielson_type_1999}.

Also, we intend to implement several improvements to the REPL, which we expect to have great impact in the programming experience.
In the first place, we want to propagate several of the features of the basic editor to the REPL, such as content assistance.
Then, we would like to enable code modifications while a program is running in the REPL.
%Still, to be effective for novices, we require to develop a strategy to impact changes in code that might have been executed and will not be executed again, such as a variable initializer: in a naïve implementation, changing it would not affect an already instantiated object.
%This kind of \emph{hot} changes can confuse even experienced programmers, so we think it is imperative to think of a specific strategy which beginners can take advantage of.
% ... referencia a scala Worksheets. https://github.com/scala-ide/scala-worksheet/wiki/Getting-Started
After a reload, the REPL should re-execute all the previous expressions in the REPL and display the new results (\cf Scala Worksheets\footnote{\url{https://github.com/scala-ide/scala-worksheet}}).
Finally, we would like an automatic conversion from a REPL session to a suite of tests.

Another characteristic of programming in the real world is the need to work in teams. 
The success of object-oriented languages is partly due to their advantages in group projects. 
It is necessary to teach our students about the techniques needed for teamwork, right from the beginning. 
To do this, it is essential that the environment has some form of support for group work \cite{kolling_problem_1999}.
Therefore, we plan to create simplified tools to integrate Wollok with \emph{version control systems}.

Also there are some initiatives to build web-based or lighter versions of the IDE and the interpreter.
This will allow Wollok to be used in context where the availability of powerful computers is restricted.
%There exists a (limited) Wollok web editor integrated to the Mumuki\footnote{\url{http://mumuki.io/}} platform, which has not been yet fully tested with students.
