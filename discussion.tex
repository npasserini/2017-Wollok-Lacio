\section{Discussion}
\label{sec:discussion}

% Discussion of actual solution \emph{vs.} initial constraints from \ref{sec:problem}. Explain the space of the solution, why we made it this way.
% Evaluation of the solution. How does the solution meet the criteria? Where does it succeed or fails...

% Hacer o no hacer un lenguaje nuevo.
A common point of controversy is whether is is worth to create a brand new language and toolset, 
instead of building our pedagogical ideas on top of existing ones, such as Self, Ruby, Smalltalk or even Eiffel.
In our experience, begginning programmers require different features from their working environment that advanced programmers
and the right selection of tools and concepts can produce substantial improvements in the learning process.
Therefore, we believe that the possibility of fine tuning provided by a specialized environment largely pays for the additional effort.

Each semester, a group of more than 20 teachers in 3 different universities share their experience with the language and tools and discuss about new features and changes to the system. 
Every modification is guided by a shared understanding about how to teach OOP \cite{lombardi_instances_2007,lombardi_carlos_alumnos_2008,griggio_programming_2011,spigariol_lucas_ensenando_2013}
% y que muestran las grandes posibilidades que se dan a partir de esta decisión inicial: imports, tests y manejo de propiedades.

A good example about teaching-specific language-design decisions is Wollok import system,
\ie the way that a programming language allows the programmer to refer in one unit of code (for example a file) to program entities defined elsewhere.
The import system allows the student to write his first very simple programs without knowing about packages or modularization, which are far too complex for him at the beginning. Still, later in the course modularization concepts are introduced and even the language forces the student to separate his code in different units. 
A full description of how the import system works and other syntax decisions can be found in \cite{javier_fernandes_wollok_2014}.

\bigskip

