\section{Classroom Experience with wollok}
\label{experience}

Since March 2015 Wollok is being used on the Universidad Nacional de San Martín. 
In 2016 the Universidad Nacional de Quilmes and Universidad Tecnológica Nacional also started using Wollok in their introductory courses.
In 2017 Wollok started being used also in a highschool. \np{Pendiente poner bien el nombre y confirmar con Fede Aloi.}

All together, it has been used in more than 15 introductory OOP courses and by more thant 500 students.
Also, many of the ideas in Wollok where present in other tools we had previously developed, going back to 2006. In these 12 years, our ideas, however evolving, have been applied by more than 100 teachers and 6000 students. 
In several cases, approval rates changed from 25\%--30\% to 80\%-90\%.
%, helping educate professionals with world-class performances both in the industrial as in the academic field. \np{No sé, suena muy agrandado?}

% Lo simple que les resulto a los chicos trabajar con una IDE
Let us describe the first experience of a new student with Wollok.
At the very beginning, the student starts by interacting with just two windows: the code editor, that points to an initial source file, and the REPL.
All the coding is done inside the editor, that includes the features of highlighting, autocompletion, and error detection/correction we described in Section \ref{sec:detectMistakes}. The streamlined WKO syntax of Wollok allows to complete the definition of simple objects with a minimum of elements. A menu option allows to access to the REPL, where the defined WKOs can be accesed by their global names, so that it is straightforward to interact with them. In the interaction through the REPL, void methods are easily distinguished from those returning a value. Furthermore, typing just an object name results in a simplified internal view, that displays the attributes along with their values.

Further on, the language and IDE features permit a gentle introduction to each successive concept we work with in the course. 
In particular, the absence of explicit type information allows for a simple implementation of polymorphism between WKOs: it suffices to have different objects that define methods for the same message.
We also remark the similarity of the syntax for WKOs and classes, described in Section Y, consent a smooth transition to the latter, motivated by the natural desire of having multiple object that share the same definition, having at the same time separate states.

While introducing design patterns is not a part of the initial OOP subject, we observe that some patterns appear naturally as we evolve to more complex problem statements. E.g. the idea of Singleton is natural in  Wollok as classes and globally accesible WKOs can be combined in the same program. Moreover, short class/object definitions can be easily added to an existing Wollok program, leading to the creation of little objects that provide specific behaviours. We note that the features of Wollok favor the reification of concepts that are not linked with the state that has to be modeled or handled. Sometimes the resulting objects follow, e.g., the Strategy or State\cite{Gamm93b} patterns.

