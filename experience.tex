\section{Classroom Experience with wollok}
\label{experience}

Since March 2015 Wollok is being used on the Universidad Nacional de San Martín, on a subject named Algoritmos 1, where the main purpose is to introduce OOP to students with minimal experience on structured programming.
In this sections we tell our experience using for the first time, this new language to teach OOP concepts.

%Lo simple que les resulto a los chicos trabajar con una IDE
First of all, lets  describe how the introduction of the students to a new IDE was, since we have been using Ozono with the same aim we propose with Wollok. 



In this case we have to separate students in two groups,  those who are having their first experience with the subject, and those who are in their second.
For the first group, we can ensure it is very easy for the them to work inside an IDE, with just two windows where you write code: one to code  objects and classes, and the other to test the objects behavior (the Repl). On the last version

For the second group, the experience is still more exciting, given that, compared with Ozono, now they have just one place to code the object. Just one file. All the object definition is in the same place. At a glance.


%La simple migración de objetos a clases.
Writing the definition of a class is not very different from writing the definition of an object, the difference is in the concept. So, we first explain to them what an object is, we write the definition of it, we see how it works. Once the students are familiar with objects and the way in which they relate with each other, we move to show how to define objects using classes, so we can create multiple objects, with the same definition, and different state.

%Los patrones de diseño que aparecieron que son muy modernos, con el uso de objetos. Era muy simple ver que usaban objetos donde antes se usaban value objects.
Some design patterns just appeared. Modelling with design patterns is not an essential part of our subject, we just introduce them to OO paradigm. But we have seen how some design pattern just appear, resulting very natural for the students. For example, having a single object with some information that more objects need (Singleton), since woollk named objects are globally are accessible\footnote{accessibility rules are still under discussion \cf Sec.\ref{sec:discussion}}. Also, since defining classes and objects in Wollok is very succint, it is frequent the emergence of small objects that configure quiet complex colaboration patterns such as Strategies or States\cite{Gamm93b}.


%El uso del REPl, aunque era una basura, fue muy bueno y los pibes le sacaron el gusto.

