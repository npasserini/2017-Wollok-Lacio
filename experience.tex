\section{Classroom Experience with wollok}
\label{experience}

Since March 2015 Wollok is being used on the Universidad Nacional de San Martín, on Algoritmos 1 subject, where the main purpose is to introduce OOP to students with minimal experience on structure programming.
In this sections we tell our experience using by the first time, this new language to learn OOP concepts.

%Lo simple que les resulto a los chicos trabajar con una IDE
First of all, lets  describe the introduction of the students to a new IDE, since they have been using Ozono with the same aim we propose with Wollok. In this case we have to separate students in two groups,  those who have their first experience with the subject, and those who are in their second.
For the first group, we can ensure it is very easy for the students to work inside an IDE, with just two windows where you write code: one to code the object, and the other to test the object behaviour (the Repl). There is nothing more to explain.

For the second group, the experience is still more exciting, given that, compared with Ozone, now they have just one place to code the object. Just one file. All the object definition is in the same place. At a glance.


%La simple migración de objetos a clases.
%Los patrones de diseño que aparecieron que son muy modernos, con el uso de objetos. Era muy simple ver que usaban objetos donde antes se usaban value objects.

%El uso del REPl, aunque era una basura, fue muy bueno y los pibes le sacaron el gusto.

