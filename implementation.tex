\section{Implementation}
\label{sec:implementation}

The way wollok is implemented is essential to enable us to build a fully customized educational language with a industrial-level toolset.
After many years of experience in the Ozono project and its predecessor, we arrived to the conclussion that
the limitations of a language implemented as an embedded DSL produce hindrances in the learning process (\cf Sec. \ref{sec:newLanguage}).
Still, giving up the embedded implementation strategy is not conceivable if we have to create all the required tools "from scratch".
Also we require a flexible implementation that allows the language to evolve, enabling and supporting our research activities.

The current implementation of Wollok language is built on top of Xtext\footnote{http://www.eclipse.org/Xtext/}, 
which is an Eclipse\footnote{https://www.eclipse.org/home/index.php}-based Language Workbench\cite{fowler2005language}.
By providing a set of tools for language development, the Xtext workbench allows us to get rid of some the necessary effort required to build a language and IDE.
Starting from a grammar defined as an extended BNF, the workbench provides several parts of the required infrastructure: 
a parser, an object-oriented AST representation, an editor capable of showing errors, basic content-assist (\figref{codetemplates.png}) and cross-references search, 
and other tools attached to the IDE, such as structured views of the code.
Also, it gives us support for implementing more advanced tools such as quick-fixes (\figref{quickfix.png}), refactorings and a debugger.
The IDE is integrated into the Eclipse platform, which in turn also helped us integrating with other Eclipse tools, such as the JUnit test runner.

\begin{figure}[ht]
    \centering
	\includegraphics[scale=0.5]{images/wollok-paper-codetemplates.png}
    \caption{Code Assist: code templates for easy edition}
    \label{fig:codetemplates.png}
\end{figure}

\begin{figure}[ht]
    \centering
	\includegraphics[scale=0.5]{images/wollok-paper-quickfix.png}
    \caption{Quick fix tool for common errors and mistakes}
    \label{fig:quickfix.png}
\end{figure}

% Interpreter
\subsection{Backend}
Given our Wollok grammar in a pseudo BNF form, Xtext then provides us with the parser and AST represented as Java ECore models part of the EMF project\footnote{http://www.eclipse.org/modeling/emf/}. 
Then there are several options for the backend when working with XText. The three most classics are: (1) to generate code through templates, (2) to generate Java code through an specific API, (3) to make your own way with a Java interpreter.

All three have their pro’s and con’s. For Wollok we decided to go on with option 3, meaning that Wollok is a fully interpreted language, with its backend being a Java application.
We prioritized simplicity for developers, \eg avoiding java code generation, and also full control of the execution in backend. 

The price for this is that we’ve lost the \emph{out-of-the-box} type checking code generation through API would give us, and also the XText debugger.
Although the first one wouldn’t have applied anyway without some effort for
customizing XText, since Wollok is a completely type annotations-free language,
and XText bases its default type system on some sort of annotations.

For the debugger, we have just developed a custom eclipse plugin which already
provides around 80\% of any industrial type language debugger.

This debugger can even be abstracted into a reusable XText component for any other interpreted language.

\subsection{Type System}
XText type system is currently implemented on top of XSemantics\footnote{http://xsemantics.sourceforge.net}, a DSL for writing rules for XText languages. XSemantics is itself made with XText.
This allows us to develop our type system in a declarative way. The type system can be seen in action in \figref{check-messageSending.png}.

\subsection{Development}
As with any software testing is really important, but unlike any common application, testing a language has many difficulties.
Wollok mixes up several techniques for testing its development.
Part of the interpreter logic is being tested with unit test in JUnit. While for some other aspects that are more tied up with xtext and eclipse components we are using XPect\footnote{http://www.xpect-tests.org}, a unit and integration-testing framework for Xtext languages.
XPect is in turn also developed as an XText language.
It provides a declarative way to annotate Wollok programs with expected
behaviour like validator’s errors/warnings, or code completions,  etc.

\section{The Wollok Software}
\seclabel{wollokSoftware}

\subsection{Features}
The Wollok Development Environment provides a number of feature for a rich programming experience, like:

\begin{itemize}
 \item Text editor with syntax highlight.
 \item Content assist
 \item Code templates
 \item Auto-build and checks along with quick fixes.
 \item Problems and warning (markers) tracking both in editors, files, and view problems. See section \secref{ChecksAndValidations}
 \item Run and debug integration with eclipse launcher framework. Providing a debug perspective integration, as well as a console.
 \item Code navigation and cross-references searches working almost as an explicitly typed language.
 \item UI Wizards for creating projects other Wollok entities.
\end{itemize}


\begin{figure}[ht]
    \centering
	\includegraphics[scale=0.5]{images/wollok-paper-outline.png}
    \caption{Outline View: This view shows the structure of the file.}
    \label{fig:outline.png}
\end{figure}

\begin{figure}[ht]
    \centering
	\includegraphics[scale=0.5]{images/wollok-paper-check-problemsview.png}
    \caption{Problems View: shows the different problems detected by the IDE }
    \label{fig:problemsview.png}
\end{figure}

\begin{figure}[ht]
    \centering
	\includegraphics[scale=0.5]{images/wollok-paper-check-noMethodOnThis.png}
    \caption{Detection of an error on sending a message to \emph{this}}
    \label{fig:check-noMethodOnThis.png}
\end{figure}


\begin{figure}[ht]
    \centering
	\includegraphics[scale=0.5]{images/wollok-paper-check-unusedVariable.png}
    \caption{Detection of unused variables}
    \label{fig:check-unusedVariable.png}
\end{figure}

\begin{figure}[ht]
    \centering
	\includegraphics[scale=0.5]{images/wollok-paper-check-messageSending.png}
    \caption{Type system in action, detecting not defined method for the message sent}
    \label{fig:check-messageSending.png}
\end{figure}

\subsection{Software availability}
Wollok is open-sourced and distributed under LGPLv3 License\footnote{http://www.gnu.org/copyleft/lgpl.html}.
Source code and documentation can be found in Bitbucket (\url{https://bitbucket.org/uqbar-project/wollok}) 
And mirrored at Github (\url{https://github.com/uqbar-project/wollok})