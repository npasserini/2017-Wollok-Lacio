\documentclass{article}
\usepackage[T1]{fontenc} %%%key to get copy and paste for the code!
\usepackage[utf8]{inputenc} %%% to support copy and paste with accents for frnehc stuff
\usepackage{times}
\usepackage[scaled=0.85]{helvet}
\usepackage{graphicx}
\usepackage{ifthen}
\usepackage{xspace}
\usepackage{alltt}
\usepackage{latexsym}
\usepackage{url}            
\usepackage{amssymb}
\usepackage{amsfonts}
\usepackage{amsmath}
\usepackage{stmaryrd}
\usepackage{enumerate}
\usepackage{cite}
\usepackage[pdftex,colorlinks=true,pdfstartview=FitV,linkcolor=blue,citecolor=blue,urlcolor=blue]{hyperref}
\usepackage{xspace}


\newboolean{showcomments}
\setboolean{showcomments}{true}
\ifthenelse{\boolean{showcomments}}
  {\newcommand{\bnote}[2]{
	\fbox{\bfseries\sffamily\scriptsize#1}
    {\sf\small$\blacktriangleright$\textit{#2}$\blacktriangleleft$}
    % \marginpar{\fbox{\bfseries\sffamily#1}}
   }
   \newcommand{\cvsversion}{\emph{\scriptsize$-$Id: macros.tex,v 1.1.1.1 2007/02/28 13:43:36 bergel Exp $-$}}
  }
  {\newcommand{\bnote}[2]{}
   \newcommand{\cvsversion}{}
  } 


\newcommand{\here}{\bnote{***}{CONTINUE HERE}}
\newcommand{\nb}[1]{\bnote{NB}{#1}}

\newcommand{\fix}[1]{\bnote{FIX}{#1}}
%%%% add your own macros 

\newcommand{\np}[1]{\bnote{Nico}{#1}}
\newcommand{\jf}[1]{\bnote{Javi}{#1}}

\graphicspath{{figures/}}
%%% 


\newcommand{\figref}[1]{Figure~\ref{fig:#1}}
\newcommand{\figlabel}[1]{\label{fig:#1}}
\newcommand{\tabref}[1]{Table~\ref{tab:#1}}
\newcommand{\layout}[1]{#1}
\newcommand{\commented}[1]{}
\newcommand{\secref}[1]{Section \ref{sec:#1}}
\newcommand{\seclabel}[1]{\label{sec:#1}}

%\newcommand{\ct}[1]{\textsf{#1}}
\newcommand{\stCode}[1]{\textsf{#1}}
\newcommand{\stMethod}[1]{\textsf{#1}}
\newcommand{\sep}{\texttt{>>}\xspace}
\newcommand{\stAssoc}{\texttt{->}\xspace}

\newcommand{\stBar}{$\mid$}
\newcommand{\stSelector}{$\gg$}
\newcommand{\ret}{\^{}}
\newcommand{\msup}{$>$}
%\newcommand{\ret}{$\uparrow$\xspace}

\newcommand{\myparagraph}[1]{\noindent\textbf{#1.}}
\newcommand{\eg}{\emph{e.g.,}\xspace}
\newcommand{\ie}{\emph{i.e.,}\xspace}
\newcommand{\etal}{\emph{et al.,}\xspace}
\newcommand{\ct}[1]{{\textsf{#1}}\xspace}
\newcommand{\cf}{\emph{cf.}\xspace}

\newenvironment{code}
    {\begin{alltt}\sffamily}
    {\end{alltt}\normalsize}

\newcommand{\defaultScale}{0.55}
\newcommand{\pic}[3]{
   \begin{figure}[h]
   \begin{center}
   \includegraphics[scale=\defaultScale]{#1}
   \caption{#2}
   \label{#3}
   \end{center}
   \end{figure}
}

\newcommand{\twocolumnpic}[3]{
   \begin{figure*}[!ht]
   \begin{center}
   \includegraphics[scale=\defaultScale]{#1}
   \caption{#2}
   \label{#3}
   \end{center}
   \end{figure*}}

\newcommand{\infe}{$<$}
\newcommand{\supe}{$\rightarrow$\xspace}
\newcommand{\di}{$\gg$\xspace}
\newcommand{\adhoc}{\textit{ad-hoc}\xspace}

\usepackage{url}            
\makeatletter
\def\url@leostyle{%
  \@ifundefined{selectfont}{\def\UrlFont{\sf}}{\def\UrlFont{\small\sffamily}}}
\makeatother
% Now actually use the newly defined style.
\urlstyle{leo}


\lstset{ %
  backgroundcolor=\color{white},   % choose the background color; you must add \usepackage{color} or \usepackage{xcolor}
  basicstyle=\footnotesize\ttfamily,        % the size of the fonts that are used for the code
  breakatwhitespace=false,         % sets if automatic breaks should only happen at whitespace
  breaklines=true,                 % sets automatic line breaking
  captionpos=b,                    % sets the caption-position to bottom
  escapeinside={\%*}{*)},          % if you want to add LaTeX within your code
  extendedchars=true,              % lets you use non-ASCII characters; for 8-bits encodings only, does not work with UTF-8
  keepspaces=true,                 % keeps spaces in text, useful for keeping indentation of code (possibly needs columns=flexible)
  numbersep=5pt,                   % how far the line-numbers are from the code
  rulecolor=\color{black},         % if not set, the frame-color may be changed on line-breaks within not-black text (e.g. comments (green here))
  showspaces=false,                % show spaces everywhere adding particular underscores; it overrides 'showstringspaces'
  showstringspaces=false,          % underline spaces within strings only
  showtabs=false,                  % show tabs within strings adding particular underscores
  stepnumber=2,                    % the step between two line-numbers. If it's 1, each line will be numbered
  tabsize=2,                       % sets default tabsize to 2 spaces
}

\lstdefinelanguage{Wollok}{
  keywords={program, console},
  sensitive=true,
  comment=[l]{//},
  morecomment=[s]{/*}{*/},
  morestring=[b]',
  morestring=[b]"
}




\begin{document}
\title{A cool title}
\maketitle


\title{Title that Describes the Contribution that Solves a Problem}
\author{O. Thor \and C. O. Ottohr}
\date{\today}
\maketitle

\begin{abstract}
In this context...
We consider this problem P...
P is a problem because...
We propose this solution...
Our solution solves P in such and such way.
\end{abstract}


\section{Introduction}
\label{sec:intro}

% Context

% Problem
Quiero tener reglas de reescritura, para poder transformar código, 
por ejemplo para hacer refactors, para tener buscadores de más alto nivel, browsear código más fácil
simplificar todas esas tareas
dejar una API que permita hacerlo programáticamente

% Known tracks for solutions
	here you want to show that you are not an idiot not knowing what have been around

% What our solution is \ct{Set} and \ct{OrderedCollection} (so that the reader knows where the paper is going)
hay una herramienta desarrollada, tiene algunos problemas, hay un blog sobre el tema.

% Contribution of the paper
una forma sencilla de definir búsquedas y transformaciones programáticamente
búsqueda y transformación basados en ejemplos (utilizando una herramienta visual para manipular los ejemplos

% Paper structure




\section{Problem Description}
\label{sec:problem}

Context, exposed with the \textbf{most precise terms possible} (don't open
unwanted doors for the reader)


Probably set the vocabulary before to cut any misinterpretation

Constraints that influenced the solution (because the solution is not
universal) \emph{e.g.} our requirements for a solution, possibly not all
satisfied. They should be sound and believable. Analysis of the criteria.
Imagine that you are another guy having this problem do the constraint
matches yours so that you could apply the solution

% Problem
no hay una forma fácil de hacer transformaciones y búsquedas genéricas orientadas a un ejémplo

The idea is very simple:
We want to change some code in a programmatic way
In order to do this, first we want to match the code we want and then change it.

Many times we want to match an expression having different information: about the selector, about the source code (temporary vars, statements).

Some examples:
all the methods with 2 arguments,
all methods called in a particular way,
a method containing a specific statement,
The first conclusion is that we want a flexible way for matching.

% Factual solution tracks, to position...

% Our solution in a nutshell.
In order to do transformations sometimes we want to refer to the existing code, sometimes this expressions are very complex.
My goal for this time is to develop tools to make simple writing searching patterns and easier writing transformations.
Here I will write about my GSoc project: Better rewriting rule tools implemented in Pharo Smalltalk, about the experience and progress.


\section{Proposed Solution}
\label{sec:contribution}
Free form, variable number of sections, technical details.

But in general do not mix solution and discussions/possible variation
let that for discussion

% Acá se podría incluir el material de http://pharorwrules.wordpress.com/2013/06/09/a-look-into-the-rewrite-tool/


\section{Discussion}
\label{sec:discussion}

Discussion of actual solution \emph{vs.} initial constraints from
\ref{sec:problem}. Explain the space of the solution, why we made it this way.

Evaluation of the solution. How does the solution meet the criteria? Where
does it succeed or fails...


\section{Related Works}
\label{sec:related}

% Other solutions in the domain, and a real comparison of our contribution with solutions from other people.

Cosas que se podrían leer:
Unterholzner \cite{Unte12a} propone otra estrategia para mejorar los refactorings, agregando static type inference.

Uquillas Gómez \etal \cite{Uqui12a}
% Verónica Uquillas Gómez, Stéphane Ducasse, Theo D'Hondt, Ring: A unifying meta-model and infrastructure for Smalltalk source code analysis tools, Computer Languages, Systems & Structures, 2012, 38, 1, 44

Bergel \etal \cite{Berg08c} 
% Alexandre Bergel, Stéphane Ducasse, Colin Putney, Roel Wuyts, Creating sophisticated development tools with OmniBrowser, Computer Languages, Systems & Structures, 2008, 34, 2-3, 109

% Tom Mens, Gabriele Taentzer, Olga Runge, Analysing refactoring dependencies using graph transformation, Software & Systems Modeling, 2007, 6, 3, 269
% Berthold Hoffmann, Dirk Janssens, Niels Van Eetvelde, Cloning and Expanding Graph Transformation Rules for Refactoring, Electronic Notes in Theoretical Computer Science, 2006, 152, 53
% Tom Mens, Niels Van Eetvelde, Serge Demeyer, Dirk Janssens, Formalizing refactorings with graph transformations, Journal of Software Maintenance and Evolution: Research and Practice, 2005, 17, 4

% Este hay que leerlo sí o sí
Eelco Visser, A survey of strategies in rule-based program transformation systems, Journal of Symbolic Computation, 2005, 40, 1, 831


% Dudosos
% Gabriele Bavota, Andrea De Lucia, Rocco Oliveto, Identifying Extract Class refactoring opportunities using structural and semantic cohesion measures, Journal of Systems and Software, 2011, 84, 3, 397
% Barbara Weber, Manfred Reichert, Jan Mendling, Hajo A. Reijers, Refactoring large process model repositories, Computers in Industry, 2011, 62, 5, 467
% Tom Mens, Serge Demeyer, Bart Du Bois, Hans Stenten, Pieter Van Gorp, Refactoring: Current Research and Future Trends, Electronic Notes in Theoretical Computer Science, 2003, 82, 3, 483

% Citar pero no necesariamente related work
% Stéphane Ducasse, Tudor Girba, Adrian Kuhn, Lukas Renggli, Meta-environment and executable meta-language using smalltalk: an experience report, Software & Systems Modeling, 2009, 8, 1, 5

\section{Conclusion}
\label{sec:conclusion}

In this paper, we looked at problem P with this context and these
constraints. We proposed solution S. It has such good points and such not so
good ones. Now we could do this or that.

\subsection*{Acknowledgements} This work was supported by Ministry of Higher Education and Research, Nord-Pas de Calais Regional Council, FEDER through the 'Contrat de
Projets Etat Region (CPER) 2007-2013',  the Cutter ANR project, ANR-10-BLAN-0219 and the MEALS Marie Curie Actions program FP7-PEOPLE-2011-
IRSES MEALS (no. 295261). 

% \bibliographystyle{plain}
% \bibliography{foo.bib}

% \appendix
% 
% \section{Lots of Furry Technical Details}

\bibliographystyle{abbrv}
\bibliography{flamel,scg}
\end{document}

%%% Local Variables: 
%%% coding: utf-8
%%% mode: latex
%%% TeX-master: "main"
%%% TeX-PDF-mode: t
%%% End:
