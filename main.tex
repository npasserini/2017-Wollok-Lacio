\documentclass[preprint,10pt]{sigplanconf}
% \documentclass{article}
\usepackage[T1]{fontenc} %%%key to get copy and paste for the code!
\usepackage[utf8]{inputenc} %%% to support copy and paste with accents for frnehc stuff
\usepackage{times}
\usepackage[scaled=0.85]{helvet}
\usepackage{graphicx}
\usepackage{ifthen}
\usepackage{xspace}
\usepackage{alltt}
\usepackage{latexsym}
\usepackage{url}            
\usepackage{amssymb}
\usepackage{amsfonts}
\usepackage{amsmath}
\usepackage{stmaryrd}
\usepackage{enumerate}

\usepackage[pdftex,colorlinks=true,pdfstartview=FitV,linkcolor=blue,citecolor=blue,urlcolor=blue]{hyperref}
\usepackage{xspace}
\usepackage{listings}


\newboolean{showcomments}
\setboolean{showcomments}{true}
\ifthenelse{\boolean{showcomments}}
  {\newcommand{\bnote}[2]{
	\fbox{\bfseries\sffamily\scriptsize#1}
    {\sf\small$\blacktriangleright$\textit{#2}$\blacktriangleleft$}
    % \marginpar{\fbox{\bfseries\sffamily#1}}
   }
   \newcommand{\cvsversion}{\emph{\scriptsize$-$Id: macros.tex,v 1.1.1.1 2007/02/28 13:43:36 bergel Exp $-$}}
  }
  {\newcommand{\bnote}[2]{}
   \newcommand{\cvsversion}{}
  } 


\newcommand{\here}{\bnote{***}{CONTINUE HERE}}
\newcommand{\nb}[1]{\bnote{NB}{#1}}

\newcommand{\fix}[1]{\bnote{FIX}{#1}}
%%%% add your own macros 

\newcommand{\np}[1]{\bnote{Nico}{#1}}
\newcommand{\jf}[1]{\bnote{Javi}{#1}}

\graphicspath{{figures/}}
%%% 


\newcommand{\figref}[1]{Figure~\ref{fig:#1}}
\newcommand{\figlabel}[1]{\label{fig:#1}}
\newcommand{\tabref}[1]{Table~\ref{tab:#1}}
\newcommand{\layout}[1]{#1}
\newcommand{\commented}[1]{}
\newcommand{\secref}[1]{Section \ref{sec:#1}}
\newcommand{\seclabel}[1]{\label{sec:#1}}

%\newcommand{\ct}[1]{\textsf{#1}}
\newcommand{\stCode}[1]{\textsf{#1}}
\newcommand{\stMethod}[1]{\textsf{#1}}
\newcommand{\sep}{\texttt{>>}\xspace}
\newcommand{\stAssoc}{\texttt{->}\xspace}

\newcommand{\stBar}{$\mid$}
\newcommand{\stSelector}{$\gg$}
\newcommand{\ret}{\^{}}
\newcommand{\msup}{$>$}
%\newcommand{\ret}{$\uparrow$\xspace}

\newcommand{\myparagraph}[1]{\noindent\textbf{#1.}}
\newcommand{\eg}{\emph{e.g.,}\xspace}
\newcommand{\ie}{\emph{i.e.,}\xspace}
\newcommand{\etal}{\emph{et al.,}\xspace}
\newcommand{\ct}[1]{{\textsf{#1}}\xspace}
\newcommand{\cf}{\emph{cf.}\xspace}

\newenvironment{code}
    {\begin{alltt}\sffamily}
    {\end{alltt}\normalsize}

\newcommand{\defaultScale}{0.55}
\newcommand{\pic}[3]{
   \begin{figure}[h]
   \begin{center}
   \includegraphics[scale=\defaultScale]{#1}
   \caption{#2}
   \label{#3}
   \end{center}
   \end{figure}
}

\newcommand{\twocolumnpic}[3]{
   \begin{figure*}[!ht]
   \begin{center}
   \includegraphics[scale=\defaultScale]{#1}
   \caption{#2}
   \label{#3}
   \end{center}
   \end{figure*}}

\newcommand{\infe}{$<$}
\newcommand{\supe}{$\rightarrow$\xspace}
\newcommand{\di}{$\gg$\xspace}
\newcommand{\adhoc}{\textit{ad-hoc}\xspace}

\usepackage{url}            
\makeatletter
\def\url@leostyle{%
  \@ifundefined{selectfont}{\def\UrlFont{\sf}}{\def\UrlFont{\small\sffamily}}}
\makeatother
% Now actually use the newly defined style.
\urlstyle{leo}


\lstset{ %
  backgroundcolor=\color{white},   % choose the background color; you must add \usepackage{color} or \usepackage{xcolor}
  basicstyle=\footnotesize\ttfamily,        % the size of the fonts that are used for the code
  breakatwhitespace=false,         % sets if automatic breaks should only happen at whitespace
  breaklines=true,                 % sets automatic line breaking
  captionpos=b,                    % sets the caption-position to bottom
  escapeinside={\%*}{*)},          % if you want to add LaTeX within your code
  extendedchars=true,              % lets you use non-ASCII characters; for 8-bits encodings only, does not work with UTF-8
  keepspaces=true,                 % keeps spaces in text, useful for keeping indentation of code (possibly needs columns=flexible)
  numbersep=5pt,                   % how far the line-numbers are from the code
  rulecolor=\color{black},         % if not set, the frame-color may be changed on line-breaks within not-black text (e.g. comments (green here))
  showspaces=false,                % show spaces everywhere adding particular underscores; it overrides 'showstringspaces'
  showstringspaces=false,          % underline spaces within strings only
  showtabs=false,                  % show tabs within strings adding particular underscores
  stepnumber=2,                    % the step between two line-numbers. If it's 1, each line will be numbered
  tabsize=2,                       % sets default tabsize to 2 spaces
}

\lstdefinelanguage{Wollok}{
  keywords={program, console},
  sensitive=true,
  comment=[l]{//},
  morecomment=[s]{/*}{*/},
  morestring=[b]',
  morestring=[b]"
}


\begin{document}
\title{Wollok -- Relearning How To Teach Object-Oriented Programming}
\authorinfo{Nicolás Passerini}
  {UTN -- Facultad Regional Buenos Aires \\ Universidad Nacional de Quilmes \\ Universidad Nacional de San Martín}
  {npasserini@gmail.com}
  
\authorinfo{Javier Fernandes}
  {Universidad Nacional de Quilmes \\ Universidad Nacional de San Martín}
  {javier.fernandes@gmail.com}

\date{\today}
\maketitle

\begin{abstract}
In this context...
We consider this problem P...
P is a problem because...
We propose this solution...
Our solution solves P in such and such way.
\end{abstract}

\section{Introduction}
\label{sec:intro}

% Context
% (a) hay que arrancar introduciendo nuestra visión, contando por qué enseñamos objetos de determinada manera, por qué hicimos Ozono/Wollok.

% La importancia de arrancar con objetos.
%\emph{Object-oriented programming} (OOP) has become the \textit{de facto} standard programming paradigm in industrial software development.
%Therefore, in the last years software engineering curricula have put more emphasis in object-oriented courses.

% Problem
%- aprender a programar se ha revelado como una tarea difícil.
Teaching how to program has revealed itself a difficult task~\cite{dijkstra_89a, jenkins2002difficulty}.
We have individualized three specific aspects present in many initial programming courses that hinder  the learning process: 
a complex programming language,
too many concepts needed for a first working program and
programming environment that are not conceived for the specific needs of an initial student~\cite{singh2012}.

%First, OO courses tend to focus too much on syntax and the particular characteristics of a language, instead of focusing on OOP distinctive characteristics.
%Second, many OO languages used in introductory courses do require to grasp a lot of quite abstract concepts before being able to build a first program.
%Finally, poor programming environments are used, although we are at a time where an unexperienced programmer could be making great use of the guidance a good programming environment could provide. These problems are not exclusive of OO courses, they are present in all the general programming courses.

%- múltiples iniciativas de entornos y lenguajes pedagógicos, para distintos públicos y con distintos objetivos.

\medskip 

%\section{Related Work}
%\label{sec:related}

% Other solutions in the domain, and a real comparison of our contribution with solutions from other people.

% Agregar referencia al paper de Fidel, y otros

% Base tomada del paper de ESUG 2011
In recent years a several pedagogical programming environment have arised, with diverse purposes.
Some of them make use of block-based or visual programming, 
such as Scratch \cite{malan_scratch_2007}, Etoys \cite{lee_empowering_2011} and Kodu \cite{kodu};
in our vision these tools are suitable for stimulating interest in programming and for being used in secondary education.
Others focus on a first universitary programming course, such as Gobstones \cite{lopez_nombre_2012}
\cl{ver si el artículo sobre Gobstones tiene related work}.
Finally, there are works that share our interest in a first OOP course, 
such as BlueJ \cite{bennedsen_bluej_2010} and Loop \cite{griggio_programming_2011}.\np{alguno más?}
Other educators propose to use industrial languages in introductory courses.
\cl{Por otro lado, también existen propuestas para usar lenguajes industriales en cursos iniciales de programación con objetos (citas a Meyer diciendo eso sobre Eiffel, alguno sobre Smalltalk, tal vez alguno que use Java).}

Wollok combines some characteristics that are typically present in academical environments with others that are more easily seen in industrial ones, aiming to enrich them to fulfill the specific needs of novice programmers and the proper development of a first OOP course.
Its most noticeable characteristics are:
(1) the possibility of using self-defined objects to support an introduction to the topic that postpones the introduction of the concept of class,
(2) the possibility of combining self-defined with class-based code objects in the same program,
(3) the decission of offering an IDE with edition and code management capabilities that are fine-tuned for unexperienced programmers
and (4) that both the language as the IDE share similarities with their industrial counterparts, in order to soften the later transition to the professional tools.


\medskip

% Nuestro trabajo
Our pedagogical approach follows the work of Lombardi \etal
\cite{lombardi_instances_2007,lombardi_carlos_alumnos_2008,griggio_programming_2011,spigariol_lucas_ensenando_2013,passerini2017wollok}, 
which consists on a novel path to introduce OO concepts focusing first on objects, messages and polymorphism 
while delaying the introduction of more abstract concepts,
such as classes, types or inheritance.
This way of organizing a course provides a more gentle learning curve to students and allows them to write completely working programs from the first classes.
Also, this approach gives great importance to the programming tools used in the course, 
stating that they should be carefully selected and customized, 
taking into account the specific needs of beginner programmers,
as well as the intended pedagogical view.

As a result, we conceived both a programming language and an accompanying integrated development environment (IDE) that closely follow the aforementioned pedagogical approach. 
Also, this new toolset is designed to overcome some disadvantages 
found in previous projects with similar approaches,
most noticeably related to two gaps not adecquately covered by the available tools:
(a) between the initial, simplified programming model and the classical OOP model
used in the late part of the courses, and
(b) between the experience in the classroom and the reality in (most) professional environments.
We consider that a essential part of this work is resolving the apparent contradition
between customizing language and environment for student needs
and at the same time keeping them close enough to their professional counterparts.

% What our solution is, so that the reader knows where the paper is going.
% Contribution of the paper

The main goal of this paper is to describe how the set of tools included in the Wollok IDE%
\footnote{
	\url{http://www.wollok.org/}. 
	Source code and documentation can be found in Github 
	(\url{https://github.com/uqbar-project/wollok}).
	Wollok is open-source and distributed under LGPLv3 License 
	(\url{http://www.gnu.org/copyleft/lgpl.html}).}, 
contributes to support a gentle and industry-aware introduction to OOP.

%Wollok combines some characteristics that are typically present in academical environments with others that are more easily seen in industrial ones, aiming to enrich them to fulfill the specific needs of novice programmers and the proper development of a first OOP course.
%Its most noticeable characteristics are:
%(1) the possibility of using self-defined objects to support an introduction to the topic that postpones the introduction of the concept of class,
%(2) the possibility of combining self-defined with class-based code objects in the same program,
%(3) the decission of offering an IDE with edition and code management capabilities that are fine-tuned for unexperienced programmers
%and (4) that both the language as the IDE share similarities with their industrial counterparts, in order to soften the later transition to the professional tools.

\medskip 
% Paper structure
%The rest of the paper is structured as follows. 
In \secref{problem} we present the problems of learning Object Oriented programming, and the consequences of this difficulties to the students. \secref{wollokLanguage} describes the proposed language and design goals. 
In \secref{environment} we describe the integrated development environment we have developed for Wollok and all the features it has and how they are useful for the teaching of programming skills. \secref{discussion} analyses the different design decisions we have taken.
%, while \secref{related} compares our solution with another similar approaches. 
Finally, we summarize our contributions in \secref{conclusion},
along with some possible lines of further work derived from this initial ideas. 
%As an appendix we have Sections \ref{sec:implementation}, \ref{sec:wollokSoftware} and \ref{sec:ChecksAndValidations} which give more details about the implementation.



%\section{Problem Description}
\section{Why Wollok?}
\label{sec:problem}

% Context, exposed with the \textbf{most precise terms possible} (don't open unwanted doors for the reader)
% Probably set the vocabulary before to cut any misinterpretation

%En la misma dirección, propongo una sección 2 bien cortita, donde nos limitemos a señalar los aspectos que apoyan lo que vamos a decir de Wollok. Y tal vez no todos, contar solamente 
%- lo del primer programa. Sobre esto, me gustaría reforzar que este programa de OO no tiene nada, con lo cual estamos desviando el foco. Se puede meter la clase Golondrina, y un main que cree una golondrina, la haga comer, y le pregunte la energía. Y fijate todo lo que hay que hacer para tener un primer ejemplo "posta-OO". Incluiría en la lista a la consola, porque no haría ejemplos con consola en Wollok.
% Los cursos se enfocan en sintaxis y usan lenguajes inadecuados.
One cause behind the difficulties in learning OOP is the use of industrial languages, which require the student to understand several concepts before being able to run his first program \cite{kolling_problem_1999}.
% Ejemplo con Java.
Figure \ref{fig:helloWorld} shows an example of a possible first program, written in Java \cite{arnold_java_1996}.
To get this program running, the student has to walk through a minefield of complex concepts: packages, classes, scoping, types, arrays, printing to standard output and class methods; just to have a first object and send a message to it.

\vspace{-3mm}
\begin{figure}[ht]
 \centering
 \begin{lstlisting}[language=Java]
	package examples;
	
	public class Accumulator {
		private int total = 0;
		
		public int getCurrentTotal() { return total; }
		public void add(amount) { total += amount; }

		public static void main(String[] args) {
			Accumulator accum = new Accumulator();
			accum.add(2);
			accum.add(5);
			accum.add(8);
			System.out.println(accum.getCurrentTotal());
		}
	}\end{lstlisting}
\vspace{-3mm}
\caption{\small Sample initial Java program which diverts student attention from the most important concepts.}
\label{fig:helloWorld}
\end{figure}

%- en particular, el ruido que le hace a los alumnos arrancar con clases.
% TODO, no estoy seguro de cómo encararlo ni de si es lo más importante.

% Por eso los pibes no aprenden
%- la tensión entre que te vaya bien en la materia y el uso de las ideas en la industria.
% TODO Iría por acá?
Courses tend to spend too much time concentrated on the details of programming constructs of a specific language, leaving too little time to become fluent on the distinctive characteristics of OOP. 
%such as identifying objects and their knowledge \emph{relationships}, assigning \emph{responsibilities} 
%and taking advantage of \emph{encapsulation} and \emph{polymorphism} to make programs more robust and extensible.
% Además necesitamos environments
Moreover, frequently the students do not have proper tools that could help them to overcome all the obstacles.
Hence we advocate the use of a pedagogically conceived programming language that allows to build simple programs from a \emph{minimum} of concepts, along with a programming environment specifically tailored for the needs of novice programmers.
%This might not be a problem for other introductory courses focused on the development of algorithms in procedural or functional languages, 
%but it has a significative importance for object-oriented courses where we want to deal with larger programs in multiple files and to teach concepts such as testing, debugging and code reuse~\cite{kolling_problem_1999}. 

\medskip

% Factual solution tracks, to position...
\np{Contar algo de otras propuestas anteriores a la nuestra}
\pt{No se si pondria algo, ya estan en los trabajos anteriores que estan citando. Podemos decir que lean el tuyo con carlono.}

% Our solution in a nutshell.
% (c) las ideas nuevas... y wollok como herramienta para dar soporte a esas ideas.

\np{A lo que sigue hasta el final de la seeción le falta una pasada.}

We have also detected that sometimes, students who seem to understand the main concepts and can apply them in interesting ways to create medium to complex program have a hard time translating this knowledge to their professional activity.
We think that bringing the activities in the course as close as possible to professional practice could help mitigate this problem.
%a good mitigation plan for this problem starts with 
For that matter, we aim to incorporate industrial practices such as 
%code repositories and 
unit tests, adapting them to the possibilities of students with little or no programming experience.

\medskip
% Our solution: environment
%The renewed approach is supported with a new programming language, named Wollok, and a programming environment which aids students to write, test and run programs.
Wollok is designed to give support to our pedagogical approach: 
it allows to define both classes and standalone objects, 
includes an industrial-like, simplified IDE, 
and 
%incorporates a basic type inferer and 
provides a simple syntax to define unit tests as well as a graphical interface to run them.

% Contribution
While neither the language itself nor the programming environment contain novel features that are unseen in industrial tools,
the assemblage of selected features, each one carefully selected due to its educational value,
is not found in other previous programming environments, neither educational nor industrial.
Therefore, the distinctive characteristic of our solution is the search for a programming toolset which 
(a) supports our pedagogical approach,
(b) feeds the student with a set of tools which are adequate to his current knowledge
and (c) gently prepares him to be using industrial-level tools. This approach constitutes a novel way of dealing with the problems of OOP teaching.

A big amount of effort in our research has been put in looking for solutions that can solve the apparent controversy between the objectives (b) and (c).
Often, the rich set of tools an industrial language or programming environment offers cannot be exploited by an inexperienced programmer or even worst they confuse the inexperienced student.
On the other hand, we think that poor programming environments fail to help students to make their first steps in programming, which in turn trims the possibilities of introductory courses.
Therefore, there is much to gain from a language that has the exact features a teacher desires to teach
and a programming environment which provides the exact tools a student can take advantage of at each time of his learning process.

% Constraints that influenced the solution (because the solution is not
% universal) \emph{e.g.} our requirements for a solution, possibly not all
% satisfied. They should be sound and believable. Analysis of the criteria.
% Imagine that you are another guy having this problem do the constraint
% matches yours so that you could apply the solution
%The current study and development have been focused on university students which have had a previous subject on imperative programming.
%The natural extension of this work is the adaptation of these ideas to teenagers or, more generally, students without any prior programming experience.


\section{A Customized Programming Environment}
\label{sec:environment}

Beginner programmers are likely to require more guidance and make more mistakes than experienced programmers.
Therefore, we think that is much to gain from a good programming environment which structures the programming experienced and helps the students to identify common mistakes.

% \subsubsection{Visualización y Navegación}
% \begin{itemize}
%   \item syntax highlight
%   \item outline
%   \item hovering
%   \item vista de problems 
%   \item navigate: goto (F3, click), flechita para ir al método que sobrescribe.
%   \item find references 
% \end{itemize}
% 
% \subsubsection{Asistencia}
% \begin{itemize}
%   \item content assist\appendix

\section{Examples of Checks and validations}
\seclabel{ChecksAndValidations}

All the results of the checking and the validation of the program is shown in one integrated view, it is called \emph{Problems}. The figure \figref{problemsview.png} shows a view of this feature. 
There are different types of problems, because these checks and validations are not only used to show type errors or syntax errors, but also to encourage some properties of the program we consider as main topics in the learning process of an OO language.

Here there is a list of all the validations and checking the tool is performing, and a brief reason why they are useful in the teaching of an object oriented language.

Todos los checkeos y problemas
generados se muestran agrupados en una vista dedicada a tal fín (Problems).

% En lo que sigue fui comentando las cosas que ya están dichas pero no quiero borrar esta enumeración porque está muy buena.
\begin{itemize}
   \item \textbf{De sintaxis}: dados por el parser y lexer automáticamente.
  \item \textbf{De estilo}: para promover uniformidad y consistencia de código.
  Ejemplos:
  		\begin{itemize}
  			\item \textit{Nombres}: variables camelcase comenzando en minúscula,
  			nombres de clases camelcase iniciando mayúscula, packages en minúsculas, etc.
  			\item \textit{Orden y agrupamiento}: dentro de un objeto o clase, primero
  			se definen sus referencias internas, luego constructores y finalmente los métodos.
  			\item \textit{Separación de programas}: las clases sólo se pueden definir
  			en archivos de tipo \textit{librería}, no dentro de un \textit{program}.
  			\item \textit{Evitar referencias duplicadas}: no se puede definir una
  			referencia con nombre ya utilizado en alguna otra referencia del contexto (local, método,
  			clase/objeto, etc.). Ni tampoco si ya está definida en la superclase.
		\end{itemize}
  \item \textbf{De resolución de referencias}: para evitar referencias a
  variables inexistentes y, en la medida de lo posible (por ser de tipado
  implícito) de envío de mensajes. Ejemplos:
  		\begin{itemize}
		  \item \textit{Referencias inexistentes}: a variables locales, parámetros, o
		  internas (clase/objeto).
		  \item \textit{Constructores inexistentes}: evaluando existencia de la
		  clase, y compatibilidad en el número de paråmetros.
		  \item \textit{Envío de mensajes (a this)}: al ser a this se pueden realizar
		  checkeos por la existencia del método y compatibilidad de parámetros, incluso sin
		  involucrar al sistema de tipos.
		\end{itemize}
  \item \textbf{De uso de referencias}: para la detección de código
  	erroneo o bien desactualizado. Por ejemplo: warnings por referencias nunca
 	utilizadas, nunca asignadas, o utilización de variables en lugar de valores.
  \item \textbf{De estructura}: evitan por ejemplo inconsistencias en las
  estructuras creadas por el alumno. Por ejemplo, se checkea
  que un método marcado como \textit{override} efectivamente esté
	sobrescribiendo.
  \item \textbf{De tipos}: verifican compatibilidad de referencias en base a sus
  tipos. Por ejemplo ante envío de mensajes, o asignaciones de variables. Basado
  en el sistema de tipos.
\end{itemize}


\section{Implementation}
\label{sec:implementation}
\np{Qué podemos decir de esto}

\section{Images}
Imágenes y otros detalles de wollok que no entran en las 6/7 páginas del artículo

	\begin{figure}[p]
	    \centering
		\includegraphics[scale=0.5]{images/wollok-paper-outline.png}
	    \caption{Outline View: muestra un resumen del contenido del archivo}
	    \label{fig:outline.png}
	\end{figure}
	
	\begin{figure}[p]
	    \centering
		\includegraphics[scale=0.5]{images/wollok-paper-check-problemsview.png}
	    \caption{Vista de Problemas: errores y warnings}
	    \label{fig:problemsview.png}
	\end{figure}
	
	\begin{figure}[p]
	    \centering
		\includegraphics[scale=0.5]{images/wollok-paper-codetemplates.png}
	    \caption{Code Assist: templates para crear código rápidamente}
	    \label{fig:codetemplates.png}
	\end{figure}

	\begin{figure}[p]
	    \centering
		\includegraphics[scale=0.5]{images/wollok-paper-check-noMethodOnThis.png}
	    \caption{Checkeo de método inexistente en this}
	    \label{fig:check-noMethodOnThis.png}
	\end{figure}


%   \item quick fixes
%   \item code templates (nuevo)
% \end{itemize}
The Wollok programming environment includes a lot of features that provide guidance to the student.
\emph{Content assist} shows the students what are his possibilities at any moment and feeds automatically into the code the most usual constructs, 
allowing the student to concentrate less on syntax and more in the modelling of the exercise problem.
\emph{Quick-fixes} allow Wollok not only to highlight problems in the student's code but also to propose automatic solutions for some usual mistakes.
\emph{Advanced code navigation} and \emph{smart reference searches} allow the programmer to better understand the dependencies in his program.
\np{¿Se les ocurre cómo mejorar eso?}
Moreover, \emph{automatic class diagrams} provide a high level view of the program and also helps understanding.

% Detect mistakes
Also, the programming environment has many tools intended to help detecting mistakes, even while the student is writing code.
\emph{Syntax highlighting} helps identify the most simple mistakes by providing immediate feedback when something is not right. 
Moreover, the environment provides \emph{real-time highlights} for syntactic mistakes.
Finally, the \emph{type inferer} allows to detect more subtle mistakes.
All these tools allows the student to gain more control of his code, keeping him away from feeling lost, 
which is otherwise a common situation for a student walking his first steps into programming.

% Este no sé cómo ponerlo, es muy crítica al smalltalk.
% 8-reducir errores frustrantes: se cancela la edicion por tener 1 solo editor de metodo por ves (poder visualizar más que un sólo método simul), evitar errores de imagenes)

\medskip
% Type inferer
The type inferer is one of the most distinctive characteristics of the Wollok programming environment.
We think that type inference is key to a simple programming environment.
On one side, it allows to detect lots of common mistakes \emph{before running the program}:
if an object does understand a message, if a wrong argument is passed, if incompatible types are mixed or even miss-spellings.
In environments without this capability it takes more time to detect errors.
Moreover, it is not uncommon that a type mistake produces a runtime error in a place different from where the mistake was done, producing confusion.

Still, providing a type inferer for a language such as Wollok has many subtleties, which deserves an independent study \cite{type inferer}.
On one side we require it to be able to work without type annotations and at the same time provide feedback useful for an inexperienced programmer.
On the other side, the type system is rather complex;
for example, the presence of stand-alone objects requires the type system to handle \emph{structural types}, since a named type system would not allow them to be treated polymorphically.
Also, we want to be able to treat polymorphically stand alone objects with class-based objects.

\subsubsection{Checks and Validations}

The IDE provides a way of showing some interesting checks and validations. This validations are organized and showed in a common way, using a dedicated section of the user interface for their display. In this way the IDE can be used to teach main concepts, not only showing syntax errors. Some of the errors are described in the \secref{ChecksAndValidations}


\section{Discussion}
\label{sec:discussion}

% Discussion of actual solution \emph{vs.} initial constraints from \ref{sec:problem}. Explain the space of the solution, why we made it this way.

This work builds on the experience of eight years using our approach in 4 different universities in Argentina\footnote{Universidad Tecnológica Nacional, F.R. Buenos Aires and F.R. Delta, 
Universidad Nacional de Quilmes, Universidad Nacional de San Martín and Universidad Nacional del Oeste.}.


% Evaluation of the solution. How does the solution meet the criteria? Where does it succeed or fails...

%\section{Related Work}
%\label{sec:related}

% Other solutions in the domain, and a real comparison of our contribution with solutions from other people.

% Agregar referencia al paper de Fidel, y otros

% Base tomada del paper de ESUG 2011
In recent years a several pedagogical programming environment have arised, with diverse purposes.
Some of them make use of block-based or visual programming, 
such as Scratch \cite{malan_scratch_2007}, Etoys \cite{lee_empowering_2011} and Kodu \cite{kodu};
in our vision these tools are suitable for stimulating interest in programming and for being used in secondary education.
Others focus on a first universitary programming course, such as Gobstones \cite{lopez_nombre_2012}
\cl{ver si el artículo sobre Gobstones tiene related work}.
Finally, there are works that share our interest in a first OOP course, 
such as BlueJ \cite{bennedsen_bluej_2010} and Loop \cite{griggio_programming_2011}.\np{alguno más?}
Other educators propose to use industrial languages in introductory courses.
\cl{Por otro lado, también existen propuestas para usar lenguajes industriales en cursos iniciales de programación con objetos (citas a Meyer diciendo eso sobre Eiffel, alguno sobre Smalltalk, tal vez alguno que use Java).}

Wollok combines some characteristics that are typically present in academical environments with others that are more easily seen in industrial ones, aiming to enrich them to fulfill the specific needs of novice programmers and the proper development of a first OOP course.
Its most noticeable characteristics are:
(1) the possibility of using self-defined objects to support an introduction to the topic that postpones the introduction of the concept of class,
(2) the possibility of combining self-defined with class-based code objects in the same program,
(3) the decission of offering an IDE with edition and code management capabilities that are fine-tuned for unexperienced programmers
and (4) that both the language as the IDE share similarities with their industrial counterparts, in order to soften the later transition to the professional tools.


\section{Conclusion}
\label{sec:conclusion}

% In this paper, we looked at problem P with this context and these
% constraints. We proposed solution S. It has such good points and such not so
% good ones. 

% Now we could do this or that.

Y a futuro le agregaríamos:
- Integración con la UI, que es otra línea de trabjo que estuvimos trabajando, tirar link a Hoope, en algún momento tenemos que integrar las dos ideas. También se puede decir que tomamos como base Gobstones.
- Integración fácil con SCM

Another characteristic of programming in the real world is the need to work in
teams. The success of object-oriented languages is partly due to their advantages in
group projects. Ideally, we also want to teach our students about the techniques
needed for teamwork. To do this, it is essential that the environment has some form
of support for group work. \cite{kolling_problem_1999}

- Más refactors y mejora del sistema de inferencia.
- Una versión web / versión liviana con el objetivo de poder ejecutarse en las netbooks que tienen los chicos de secundaria.

Probar la nueva herramienta en entornos educativos.

Eso pensando el lenguaje/herramienta, si pienso a nivel docencia se me ocurre que lo que tenemos que hacer es integrarnos con otras entidades, como Sadosky u otras universidades.
Haciendo foco en que el punto no es la herramienta sino que tenemos que repensar cómo enseñamos a programar.

\subsection*{Acknowledgements} 
% This work was supported by Ministry of Higher Education and Research, Nord-Pas de Calais Regional Council, FEDER through the 'Contrat de
% Projets Etat Region (CPER) 2007-2013',  the Cutter ANR project, ANR-10-BLAN-0219 and the MEALS Marie Curie Actions program FP7-PEOPLE-2011-
% IRSES MEALS (no. 295261). 

% \bibliographystyle{plain}
% \bibliography{foo.bib}

% \appendix
% 
% \section{Lots of Furry Technical Details}

{
\small
\bibliographystyle{abbrv}
\bibliography{wollok,Teaching,scg}
}

\newpage
\appendix
\section{Images}
Imágenes y otros detalles de wollok que no entran en las 6/7 páginas del artículo

	\begin{figure}[p]
	    \centering
		\includegraphics[scale=0.5]{images/wollok-paper-outline.png}
	    \caption{Outline View: muestra un resumen del contenido del archivo}
	    \label{fig:outline.png}
	\end{figure}
	
	\begin{figure}[p]
	    \centering
		\includegraphics[scale=0.5]{images/wollok-paper-check-problemsview.png}
	    \caption{Vista de Problemas: errores y warnings}
	    \label{fig:problemsview.png}
	\end{figure}
	
	\begin{figure}[p]
	    \centering
		\includegraphics[scale=0.5]{images/wollok-paper-codetemplates.png}
	    \caption{Code Assist: templates para crear código rápidamente}
	    \label{fig:codetemplates.png}
	\end{figure}

	\begin{figure}[p]
	    \centering
		\includegraphics[scale=0.5]{images/wollok-paper-check-noMethodOnThis.png}
	    \caption{Checkeo de método inexistente en this}
	    \label{fig:check-noMethodOnThis.png}
	\end{figure}


\end{document}
