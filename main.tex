\documentclass{article}
\usepackage[T1]{fontenc} %%%key to get copy and paste for the code!
\usepackage[utf8]{inputenc} %%% to support copy and paste with accents for frnehc stuff
\usepackage{times}
\usepackage[scaled=0.85]{helvet}
\usepackage{graphicx}
\usepackage{ifthen}
\usepackage{xspace}
\usepackage{alltt}
\usepackage{latexsym}
\usepackage{url}            
\usepackage{amssymb}
\usepackage{amsfonts}
\usepackage{amsmath}
\usepackage{stmaryrd}
\usepackage{enumerate}
\usepackage{cite}
\usepackage[pdftex,colorlinks=true,pdfstartview=FitV,linkcolor=blue,citecolor=blue,urlcolor=blue]{hyperref}
\usepackage{xspace}

\newboolean{showcomments}
\setboolean{showcomments}{true}
\ifthenelse{\boolean{showcomments}}
  {\newcommand{\bnote}[2]{
	\fbox{\bfseries\sffamily\scriptsize#1}
    {\sf\small$\blacktriangleright$\textit{#2}$\blacktriangleleft$}
    % \marginpar{\fbox{\bfseries\sffamily#1}}
   }
   \newcommand{\cvsversion}{\emph{\scriptsize$-$Id: macros.tex,v 1.1.1.1 2007/02/28 13:43:36 bergel Exp $-$}}
  }
  {\newcommand{\bnote}[2]{}
   \newcommand{\cvsversion}{}
  } 


\newcommand{\here}{\bnote{***}{CONTINUE HERE}}
\newcommand{\nb}[1]{\bnote{NB}{#1}}

\newcommand{\fix}[1]{\bnote{FIX}{#1}}
%%%% add your own macros 

\newcommand{\np}[1]{\bnote{Nico}{#1}}
\newcommand{\jf}[1]{\bnote{Javi}{#1}}

\graphicspath{{figures/}}
%%% 


\newcommand{\figref}[1]{Figure~\ref{fig:#1}}
\newcommand{\figlabel}[1]{\label{fig:#1}}
\newcommand{\tabref}[1]{Table~\ref{tab:#1}}
\newcommand{\layout}[1]{#1}
\newcommand{\commented}[1]{}
\newcommand{\secref}[1]{Section \ref{sec:#1}}
\newcommand{\seclabel}[1]{\label{sec:#1}}

%\newcommand{\ct}[1]{\textsf{#1}}
\newcommand{\stCode}[1]{\textsf{#1}}
\newcommand{\stMethod}[1]{\textsf{#1}}
\newcommand{\sep}{\texttt{>>}\xspace}
\newcommand{\stAssoc}{\texttt{->}\xspace}

\newcommand{\stBar}{$\mid$}
\newcommand{\stSelector}{$\gg$}
\newcommand{\ret}{\^{}}
\newcommand{\msup}{$>$}
%\newcommand{\ret}{$\uparrow$\xspace}

\newcommand{\myparagraph}[1]{\noindent\textbf{#1.}}
\newcommand{\eg}{\emph{e.g.,}\xspace}
\newcommand{\ie}{\emph{i.e.,}\xspace}
\newcommand{\etal}{\emph{et al.,}\xspace}
\newcommand{\ct}[1]{{\textsf{#1}}\xspace}
\newcommand{\cf}{\emph{cf.}\xspace}

\newenvironment{code}
    {\begin{alltt}\sffamily}
    {\end{alltt}\normalsize}

\newcommand{\defaultScale}{0.55}
\newcommand{\pic}[3]{
   \begin{figure}[h]
   \begin{center}
   \includegraphics[scale=\defaultScale]{#1}
   \caption{#2}
   \label{#3}
   \end{center}
   \end{figure}
}

\newcommand{\twocolumnpic}[3]{
   \begin{figure*}[!ht]
   \begin{center}
   \includegraphics[scale=\defaultScale]{#1}
   \caption{#2}
   \label{#3}
   \end{center}
   \end{figure*}}

\newcommand{\infe}{$<$}
\newcommand{\supe}{$\rightarrow$\xspace}
\newcommand{\di}{$\gg$\xspace}
\newcommand{\adhoc}{\textit{ad-hoc}\xspace}

\usepackage{url}            
\makeatletter
\def\url@leostyle{%
  \@ifundefined{selectfont}{\def\UrlFont{\sf}}{\def\UrlFont{\small\sffamily}}}
\makeatother
% Now actually use the newly defined style.
\urlstyle{leo}


\lstset{ %
  backgroundcolor=\color{white},   % choose the background color; you must add \usepackage{color} or \usepackage{xcolor}
  basicstyle=\footnotesize\ttfamily,        % the size of the fonts that are used for the code
  breakatwhitespace=false,         % sets if automatic breaks should only happen at whitespace
  breaklines=true,                 % sets automatic line breaking
  captionpos=b,                    % sets the caption-position to bottom
  escapeinside={\%*}{*)},          % if you want to add LaTeX within your code
  extendedchars=true,              % lets you use non-ASCII characters; for 8-bits encodings only, does not work with UTF-8
  keepspaces=true,                 % keeps spaces in text, useful for keeping indentation of code (possibly needs columns=flexible)
  numbersep=5pt,                   % how far the line-numbers are from the code
  rulecolor=\color{black},         % if not set, the frame-color may be changed on line-breaks within not-black text (e.g. comments (green here))
  showspaces=false,                % show spaces everywhere adding particular underscores; it overrides 'showstringspaces'
  showstringspaces=false,          % underline spaces within strings only
  showtabs=false,                  % show tabs within strings adding particular underscores
  stepnumber=2,                    % the step between two line-numbers. If it's 1, each line will be numbered
  tabsize=2,                       % sets default tabsize to 2 spaces
}

\lstdefinelanguage{Wollok}{
  keywords={program, console},
  sensitive=true,
  comment=[l]{//},
  morecomment=[s]{/*}{*/},
  morestring=[b]',
  morestring=[b]"
}


\begin{document}
\title{Wollok -- y una frase que describa la idea}
\author{Nicolás Passerini \and Javier Fernandes}
\date{\today}
\maketitle

\begin{abstract}
In this context...
We consider this problem P...
P is a problem because...
We propose this solution...
Our solution solves P in such and such way.
\end{abstract}


\section{Introduction}
\label{sec:intro}

% Context

(a) hay que arrancar introduciendo nuestra visión, contando por qué enseñamos objetos de determinada manera, por qué hicimos Ozono.

Enseñanza tradicional:
- Muchas veces se parte del lenguaje
- Bajos niveles de aprobación
- Pasa mucho tiempo hasta que un chico puede hacer un programa "real", demasiados conceptos.
- ... después pienso más

Nosotros propusimos:
- Elegir un recorrido que permite ir incorporando los conceptos uno a uno
- Tener una herramienta que da soporte a eso
- Focalizar en objeto-mensaje-polimorfismo-referencias, los demás conceptos aparecen después.


Also these hindrances reduce the opportunity of students to apply
the concepts of the paradigm effectively in their further
professional practice, resulting in several IT-projects not taking
advantage of the possibilities offered by the potential of good
object-oriented practices, even in cases where the tools used may
allow the application of object-oriented programming gracefully. \cite{lombardi_instances_2007}

% Problem

\medskip 

% Known tracks for solutions
We propose to provide the student a reduced and graphical
programming environment in which the object and the message are
the central concepts instead of defining classes and then instantiate
them. \cite{griggio_programming_2011}

% here you want to show that you are not an idiot not knowing what have been around


\medskip 

% What our solution is \ct{Set} and \ct{OrderedCollection} (so that the reader knows where the paper is going)

(b) lo que aprendimos en estos 8 años haciendo eso, que nos lleva a querer darle una vuelta más.

Si bien todo esto que se nos ocurrió en su momento fue maravilloso y genial y los pibes aprenden más, programan mejor y la ponen más seguido, fuimos aprendiendo más cosas y estamos trabajando ne unos cambios:

Los fáciles:
- integrar clases y objetos en un mismo programa
- integrar interfaces de usuario automáticas
- utilizar herramientas avanzadas para guiar a los alumnos en el proceso de aprendizaje, la vedette acá sería el sistema de tipos.
- acercar la práctica de lo que hacemos a la práctica industrial (acá hay que ver qué decimos y qué no, por ejemplo me molesta que no sea un archivo... pero desde lo metodológico se puede pensar en tests o incluso en un repositorio de código... sobre la relación con la industria se podría escribir un libro, hay que ver cuánto queremos meternos).

No sé si hablar de los temas de colecciones, son un poco particulares de Ozono.


\medskip 

% Contribution of the paper
(c) las ideas nuevas... y wollok como herramienta para dar soporte a esas ideas.

Lo concreto que hicimos es tener un lenguaje con
- clases y objetos integrados
- un IDE profesional con syntax highlighting, refactors, autocompletion y la vedette (?) inferencia de tipos (en progreso).
- bueno, muchas mejoras a nivel lenguaje, como literales para coleciones, imports
- una forma fácil de construir tests.

\medskip 

% Paper structure


\section{Problem Description}
\label{sec:problem}

Context, exposed with the \textbf{most precise terms possible} (don't open
unwanted doors for the reader)


Probably set the vocabulary before to cut any misinterpretation

Constraints that influenced the solution (because the solution is not
universal) \emph{e.g.} our requirements for a solution, possibly not all
satisfied. They should be sound and believable. Analysis of the criteria.
Imagine that you are another guy having this problem do the constraint
matches yours so that you could apply the solution

% Problem

% Factual solution tracks, to position...

% Our solution in a nutshell.


\section{Proposed Solution}
\label{sec:contribution}
% Free form, variable number of sections, technical details.
% But in general do not mix solution and discussions/possible variation let that for discussion

\section{Discussion}
\label{sec:discussion}

% Discussion of actual solution \emph{vs.} initial constraints from \ref{sec:problem}. Explain the space of the solution, why we made it this way.

% Evaluation of the solution. How does the solution meet the criteria? Where does it succeed or fails...


\section{Related Works}
\label{sec:related}

% Other solutions in the domain, and a real comparison of our contribution with solutions from other people.

\section{Conclusion}
\label{sec:conclusion}

% In this paper, we looked at problem P with this context and these
% constraints. We proposed solution S. It has such good points and such not so
% good ones. 

% Now we could do this or that.

Y a futuro le agregaríamos:
- Integración con la UI
- Integración fácil con SCM
- Más refactors y mejora del sistema de inferencia.
- Una versión web / versión liviana con el objetivo de poder ejecutarse en las netbooks que tienen los chicos de secundaria.

Probar la nueva herramienta en entornos educativos.

Eso pensando el lenguaje/herramienta, si pienso a nivel docencia se me ocurre que lo que tenemos que hacer es integrarnos con otras entidades, como Sadosky u otras universidades.
Haciendo foco en que el punto no es la herramienta sino que tenemos que repensar cómo enseñamos a programar.

\subsection*{Acknowledgements} 
% This work was supported by Ministry of Higher Education and Research, Nord-Pas de Calais Regional Council, FEDER through the 'Contrat de
% Projets Etat Region (CPER) 2007-2013',  the Cutter ANR project, ANR-10-BLAN-0219 and the MEALS Marie Curie Actions program FP7-PEOPLE-2011-
% IRSES MEALS (no. 295261). 

% \bibliographystyle{plain}
% \bibliography{foo.bib}

% \appendix
% 
% \section{Lots of Furry Technical Details}

\bibliographystyle{abbrv}
\bibliography{wollok,teaching,scg}
\end{document}
