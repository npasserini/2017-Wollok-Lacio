\documentclass{beamer}

\usecolortheme[named=blue]{structure}

\mode<presentation>
{
  \usetheme{Warsaw}
  \setbeamercovered{transparent}
  \setbeamertemplate{items}[ball]
  \setbeamertemplate{theorems}[numbered]
  \setbeamertemplate{footline}[frame number]
}
\usepackage{beamerthemesplit}
\usepackage{graphics}
\usepackage{graphicx}
\usepackage{hyperref}
\usepackage{color}
\usepackage{listings}
\usepackage[utf8]{inputenc}

\newcommand{\code}[1]{\texttt{\small{#1}}}
\hypersetup{%
  colorlinks=true,
  urlcolor=blue,
  linkcolor=blue,
  pdfborderstyle={/S/U/W 1}
}

\definecolor{javared}{rgb}{0.6,0,0} % for strings
\definecolor{javagreen}{rgb}{0.25,0.5,0.35} % comments
\definecolor{javapurple}{rgb}{0.5,0,0.35} % keywords
\definecolor{javadocblue}{rgb}{0.25,0.35,0.75} % javadoc

\lstdefinelanguage{scala}{
  morekeywords={abstract,case,catch,class,def,%
    do,else,extends,false,final,finally,%
    for,if,implicit,import,match,mixin,%
    new,null,object,override,package,%
    private,protected,requires,return,sealed,%
    super,this,throw,trait,true,try,%
    type,val,var,while,with,yield},
  otherkeywords={=>,<-,<\%,<:,>:,\#,@,>,<},
  sensitive=true,
  morecomment=[l]{//},
  morecomment=[n]{/*}{*/},
  morestring=[b]",
  morestring=[b]',
  morestring=[b]"""
}
\lstset{
%   frame=tb,
  language=Scala,
  aboveskip=3mm,
  belowskip=3mm,
  columns=flexible,
  basicstyle=\ttfamily\small,
  keywordstyle=\color{javapurple}\bfseries,
  stringstyle=\color{javared},
  commentstyle=\color{javagreen},
  morecomment=[s][\color{javadocblue}]{/**}{*/},
  numbers=none,
  numberstyle=\tiny\color{black},
  stepnumber=2,
  numbersep=10pt,
  showspaces=false,
  showstringspaces=false,
  tabsize=4
}

\title
  [Wollok: Relearning to teach OO programming.]
  {Wollok: Relearning to teach OO programming.}
\author[Passerini, Fernandes, Tesone]{%
  Javier Fernandes\inst{1,2} \and
  Nicolás Passerini\inst{1,2,4} \\
  Pablo Tesone\inst{3,1,2,4}
}  

\institute{
  \inst{1}Universidad Nacional de Quilmes
  \and
  \inst{2}Universidad Nacional de San Martin
  \and
  \inst{3}Universidad Nacional del Oeste
  \and
  \inst{4}Universidad Tecnológica Nacional - F.R. Buenos Aires.
}

\date[WISIT 2014]{\small Workshop de Ingeniería en Sistemas y Tecnologías de la Información \\ 28/11/2014}
\subject{Computational Sciences}

%\logo{\includegraphics[height=1.0cm]{fsu_logo.pdf}}

\begin{document}
  \frame
  {
    \titlepage
  }

  \frame
  {
    \frametitle{Agenda}
    \tableofcontents
  }

\defverbatim[colored]\helloWorldJava{
\begin{lstlisting}[language=Java]
		package examples;
		
		public class HelloWorld {
			public static void main(String[] args) {
				System.out.println("Hello World");
			}
		}
\end{lstlisting}
}

\section{Problema}
\frame{
	\frametitle{¿Por qué es difícil aprender OOP?}	
	\begin{itemize}
    \item Enfoque en un lenguaje particular
    \item Demasiados conceptos
    \medskip
    \helloWorldJava
    \pause
    \item Entornos de desarrollo pobres
	\end{itemize}
}

\frame{
	\frametitle{Conclusión: deficiencias de aprendizaje}	
	\begin{itemize}
		\item Bajos niveles de aprobación
		\item Se propician malas prácticas
		\item Poca comprensión de los fundamentos del paradigma
	\end{itemize}
}

\section{Un poco de historia: Ozono}
\frame{
	\frametitle{Nuestra primera propuesta}
	\begin{enumerate}
		\item Pensar en el recorrido\footnote{Paper de Lombardi, Passerini y Cesario, 2007}
		\begin{itemize}
			\item Introducir los conceptos gradualmente.
			\item Arrancar por los fundamentales:
				\\\hspace{2em} objeto - mensaje - referencia - polimorfismo
			\item Postergar los accesorios: 
				\\\hspace{2em} clases - herencia - ...
		\end{itemize}

		\pause
		\medskip
		\item Construir una herramienta específica
		\begin{itemize}
			\item Lenguaje de programación
			\item Entorno de desarrollo
		\end{itemize}
	\end{enumerate}
}

\frame{
	\frametitle{Ozono}	
	\begin{itemize}
		\item Entorno de objetos sin clases
		\item Basado en Pharo
		\begin{itemize}
		 \item Basado en imagen (no archivos)
		 \item Dinámico
		 \item Diagrama automático de objetos
		\end{itemize}
	\end{itemize}
}

\frame{
	\frametitle{7 años de aprendizaje}	
	Ozono fue una linda experiencia:
	\begin{itemize}
		\item Subieron los niveles de aprobación (de 40 - 50\% a 80 - 90\%)
		\item Exportado a otras universidades: UNQ, UNSAM, UNO, FRD, ...
		\item Comunidad $>$ 30 desarrolladores
		\item Proyectos de investigación		
	\end{itemize}

	\bigskip
	Peeero...
}

\frame{
	\frametitle{7 años de aprendizaje}	
	\begin{itemize}
		\item No provee una transición a clases
		\pause
		\item Falencias en el ambiente
			\begin{itemize}
				\item Muy atado a pharo (ej: Debugger)
				\item Herramientas no preparadas para el aprendizaje
			\end{itemize}
		\pause
		\item A veces... extrañamos los tipos
			\begin{itemize}
				\item Las herramientas no pueden ayudar al programador
				\item Errores simples son difíciles de detectar.
			\end{itemize}
		\pause
		\item Lejano al \emph{mainstream}
			\begin{itemize}
				\item Basado en imagen
				\item No usa herramientas estándares de la industria
			\end{itemize}
		
	\end{itemize}
}

\section{Lo que viene: Wollok}
\frame{
	\frametitle{Wollok the language}	
	\begin{itemize}
    \item Integra clases y objetos
    \item Sintaxis \emph{educativa}
			\begin{itemize}
				\item Énfasis en los conceptos a transmitir (ej: method, val/var).
				\item Se eliminan conceptos innecesarios
				\item \emph{Moderna}: literales, bloques, etc.
			\end{itemize}
    \item Inferencia de tipos
    \item Basado en archivos
	\end{itemize}
}

\frame{
	\frametitle{¿Qué debería tener un entorno de desarrollo?}	
	\begin{itemize}
		\item \emph{Autocomplete}
		\item Navegación (ej: "Ir a la implementación")
		\item Búsqueda inteligente (ej: referencias, implementaciones, usos)
		\item Wizards
		\item Refactors automáticos
		\item Herramientas de debugging
		\item Herramientas de trabajo en grupo
		\item Herramientas de visualización de código
	\end{itemize}
}

\section{Demo}
\frame{
      \frametitle{Demo}
}

\section{Conclusiones y trabajo futuro}
\frame{
	\frametitle{Conclusions}	
	\begin{itemize}
	  \item	Necesitamos herramientas y lenguajes específicos para la enseñanza.
	  \item La forma de enseñanza debe estar guiada por el público al que esta dirigido.
	  \item Repensar en el recorrido.
	  \item Elegir buenos ejemplos.
	\end{itemize}
}

\frame{
	\frametitle{Further Work}	
	\begin{itemize}
	  \item Integrar una versión colaborativa Web
	  \item Desarrollar nuevas herramientas de refactors
	  \item Integrar una interfaz gráfica interactiva.
	  \item Versiones simplificadas de manejo de código.
	  \item Pruebas en ambientes de enseñanza universitarios y secundarios.
	\end{itemize}
}

\frame{
  \frametitle{Muchas gracias}.
}
\end{document}

