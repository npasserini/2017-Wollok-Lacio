\section{Proposed Solution}
\label{sec:contribution}

% Free form, variable number of sections, technical details.
% But in general do not mix solution and discussions/possible variation let that for discussion

Wollok is a complete new tool based on the same fundamental ideas that were present in Ozono and LOOP. In particular:

\begin{itemize}
\item \textbf{Incremental concepts introduction}: references, objects, messages, polymorphism, classes, inheritance.
\item \textbf{General purpose}: meaning not tied to any specific domain (e.g.: robots)
\end{itemize}

Besides this we also tried to address some other concerns detected while using Ozono for several years. Here's a list of main cathegories or lines of work in which Wollok extends the previous work

\begin{itemize}
\item \textbf{Profundizar y pulir el highlighting the conceptos primarios y la
estratificacion de conceptos}.
	(ej: literales de objetos, literales de colecciones. Objetos no como un
	elemento de la IDE -Ozono: nueva referencia global-, sino como un elemnto del
	lenguaje. Evita referencias globales.)
\item \textbf{Introducción de nuevos elementos concretos que explicitan
conceptos ya existentes} (ej: 1- var/val, 2- la idea de hacer un effect system
power que detecte efecto de lado, y asi poner checkeos para resolver el problema de si un método es una 'orden' o una 'pregunta', 3- program/libreria/test, 4-override ).
\item \textbf{Unificar las fases del aprendizaje} (ej: objetos+clases: un solo
lenguaje, misma herramientas, poder reutilizar y hacer convivir)
\item \textbf{Proveer un entorno inteligente que}: por un lado, estructure en
forma más estricta/explícita la experiencia; y que, por el otro lado, permita una gran asistencia al estudiante/desarrollador (esto tiene muchos elementos: 1- desde content assist, 2-syntax coloring, 3- resaltado de errores (sintaxis y tipado) 4-navegación de código, 5-busqueda de referencias, 6-diagramas automáticos de clases, 7-hasta llegar un sistema de tipos que permita la detección temprana de errores, 8-reducir errores frustrantes: se cancela la edicion por tener 1 solo editor de metodo por ves (poder visualizar más que un sólo método simul), evitar errores de imagenes)
\item \textbf{Acercar la experiencia de aprendizaje a las prácticas
industriales}: (acá el palo de que la imagen sólo existe en smalltalk, y en la
industria nadie la usa. Atrás de eso, la idea de archivos, y poder compartir con SVC. Por último la idea de actualizarse a un lenguaje con influencia de lenguajes modernos como xtend, scala, ruby, etc.)
\end{itemize}