%\section{Related Work}
%\label{sec:related}

% Other solutions in the domain, and a real comparison of our contribution with solutions from other people.

% Agregar referencia al paper de Fidel, y otros

% Base tomada del paper de ESUG 2011
In recent years a several pedagogical programming environment have arised, with diverse purposes.
Some of them make use of block-based or visual programming, 
such as Scratch \cite{malan_scratch_2007}, Etoys \cite{lee_empowering_2011} and Kodu \cite{kodu};
in our vision these tools are suitable for stimulating interest in programming and for being used in secondary education.
Others focus on a first universitary programming course, such as Gobstones \cite{lopez_nombre_2012}
\cl{ver si el artículo sobre Gobstones tiene related work}.
Finally, there are other pedagogical programming tool proposals that share our interest in a first OOP course, such as BlueJ \cite{bennedsen_bluej_2010} and Loop \cite{griggio_programming_2011}.\np{alguno más?}

Other educators propose to use industrial languages in introductory courses, such as Java \cite{kolling2001guidelines}, Eiffel \cite{meyer1993towards}, Smalltalk \cite{ducasse2006squeak}
and Self \cite{Unga87a}.
Self, at the same time, has pioneered in allowing for OOP without classes.
