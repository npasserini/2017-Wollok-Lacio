\section{Related Work}
\label{sec:related}

% Other solutions in the domain, and a real comparison of our contribution with solutions from other people.

% Agregar referencia al paper de Fidel, y otros

% Base tomada del paper de ESUG 2011

The first aspect to analyse is the shape of OOP introductory courses.
Vilmer \etal \cite{vilner_2007} presents a work exposing the advantages of the implementation of object-first introductory courses. 
Also, Moritz \etal \cite{moritz_2005} presents a way of starting the learning of a programming language using an object-first way using multimedia and intelligent tutoring.
Another interesting work in this area is the one from Sajaniemi \etal \cite{Sajaniemi_teachingprogramming:} who presents another way to introduce the main concepts.
All this authors propose to use an industrial language, such as Java or C\#, but they do not address the problems arising from the use of these languages.
On the other hand, Lopez \etal \cite{lopez_nombre_2012} present a successful way of teaching using functional-first in an introductory course. 

Another aspect to analyse is the use of an industrial programming language or a custom one. 
In this subject, the approcah of Lopez \etal \cite{lopez_nombre_2012} is similar to ours, but in a functional-first solution. 
As Wollok, his language is focused on the main concepts of the paradigm.
Another custom language specifically built to focus on the main concepts of OOP is BlueJ \cite{bennedsen_bluej_2010}. 
This implementation shares with Wollok the idea to simplify the language, but it is class centered. 
Wollok is both class and object centered, so we avoid need to teach classes to start learning the basic concepts of the paradigm.

There are interesting works in the Visual languages as a way of teaching OOP: Scratch \cite{malan_scratch_2007}, Etoys \cite{lee_empowering_2011} and Kodu \cite{kodu}. 
Still, all of them are far away of a professional development environment, so the transition to a industrial level work is not so easy as with Wollok.
