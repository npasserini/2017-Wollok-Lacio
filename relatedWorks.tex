%\section{Related Work}
%\label{sec:related}

% Other solutions in the domain, and a real comparison of our contribution with solutions from other people.

% Agregar referencia al paper de Fidel, y otros

% Base tomada del paper de ESUG 2011
In recent years a several pedagogical programming environment have arised, with diverse purposes.
Some of them make use of block-based or visual programming, 
such as Scratch \cite{malan_scratch_2007}, Etoys \cite{lee_empowering_2011} and Kodu \cite{kodu};
in our vision these tools are suitable for stimulating interest in programming and for being used in secondary education.
Others focus on a first universitary programming course, such as Gobstones \cite{lopez_nombre_2012}
\cl{ver si el artículo sobre Gobstones tiene related work}.
Finally, there are works that share our interest in a first OOP course, 
such as BlueJ \cite{bennedsen_bluej_2010} and Loop \cite{griggio_programming_2011}.\np{alguno más?}
Other educators propose to use industrial languages in introductory courses.
\cl{Por otro lado, también existen propuestas para usar lenguajes industriales en cursos iniciales de programación con objetos (citas a Meyer diciendo eso sobre Eiffel, alguno sobre Smalltalk, tal vez alguno que use Java).}

Wollok combines some characteristics that are typically present in academical environments with others that are more easily seen in industrial ones, aiming to enrich them to fulfill the specific needs of novice programmers and the proper development of a first OOP course.
Its most noticeable characteristics are:
(1) the possibility of using self-defined objects to support an introduction to the topic that postpones the introduction of the concept of class,
(2) the possibility of combining self-defined with class-based code objects in the same program,
(3) the decission of offering an IDE with edition and code management capabilities that are fine-tuned for unexperienced programmers
and (4) that both the language as the IDE share similarities with their industrial counterparts, in order to soften the later transition to the professional tools.
