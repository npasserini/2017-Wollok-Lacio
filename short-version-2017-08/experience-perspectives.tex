\section{Experience and perspectives}
\label{sec:experience-perspectives}

Several hundreds of students have used Wollok in courses along several universities of the Buenos Aires metropolitan area, since 2015.
In this experience, both the language and the IDE have successfully supported the incremental learning path described in Section~\ref{sec:problem}.
The combination of the minimal syntax required to define a WKO, the quick interaction that the REPL enables, and the features included in the editor, allows to greatly shorten the time needed by a standard student to get her first object definitions work, right from the first lab session.

We also remark that:
(a) The transition from WKOs to classes was remarkably smooth.
(b) Students get used to write unit tests from the first course weeks.
(c) The tools and features bundled in the IDE effectively empower students, helping them to avoid simple syntax errors, and to quickly discover other potential sources of problemes.

In turn, the experience of some students in a later course in which Java is introduced, suggests that the industrial-like appearance of the IDE and the language syntax ease the transition into work-like environments. Hence the use of Wollok in the first course seems to promote the effective use of the basic concepts and techniques learned there, in their successive experiences.

\medskip
The main focuses of attention for the Wollok IDE development team are the detection of programming errors and bad practices, and the provision of quick fixes, content assistance and refactorings.
A cornerstone to achieve these goals is the \emph{type inferer}, which is one of our current main objectives.
Still, providing a type inferer for a language such as Wollok has many subtleties, which deserve an independent study \cite{passerini_nicolas_extensible_2014}.
Also, we plan to include an \emph{effect system} \cite{nielson_type_1999}.

On the other hand, in our opinion \emph{teamwork} should be a subject covered in programming courses from as early as possible. We note that the success of object-oriented languages is partly due to their advantages in group projects. 
Hence we plan to give support to teamwork in Wollok, by adding a simplified interface to \emph{code repositories} like Git or SVN.

There are also initiatives to build \emph{Web-based} or lighter versions of the IDE and the interpreter, in order to broaden the availability of Wollok.
%This will allow Wollok to be used in context where the availability of powerful computers is restricted.
























