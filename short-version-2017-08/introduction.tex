\section{Introduction}
\label{sec:intro}

Teaching how to program has revealed itself a difficult task~\cite{dijkstra_89a, jenkins2002difficulty}.
We have individualized three specific aspects present in many initial programming courses that hinder  the learning process: 
a complex programming language,
the need of introducing too many concepts to produce a first working program, and
%too many concepts needed for a first working program and
a programming environment not conceived for the specific needs of an initial student~\cite{singh2012}.

\medskip 

% Known tracks for solutions
% here you want to show that you are not an idiot not knowing what have been around

%- en particular en objetos hay iniciativas, tanto de presentar un modelo inicial simplificado, como de definir lenguajes y entornos de propósito pedagógico.

% Primero hablar de lenguajes específicos para enseñar
There have been several proposals to address the difficulties in introductory OO courses. Some of them define a specific language which provides a simplified programming model such as Karel++~\cite{bergin_karel++:_1996} and Mama~\cite{harrisonmama}; others even provide a whole programming environment specifically designed to aid novice programmers 
such as Squeak \cite{ingalls_back_1997}, Traffic \cite{broy_outside-method_2003} and BlueJ \cite{bennedsen_bluej_2010}. 
% Object first
The great differences between these programming languages and environments show that they have to be analysed in the light of the pedagogical approaches behind them.
The tools are of little use without their respective pedagogical view.
%For example, some educational languages and environments are designed to be used in \textit{object-first} approaches, 
%\ie for students without any previous programming knowledge \cite{arnow_introduction_1998, bruce_library_2001}.

\medskip
% Nuestro trabajo
%- en nuestra experiencia dando cursos iniciales de OOP, adoptamos un modelo simplificado inicial que prescinde de la noción de clase. Después de algunas semanas, pasamos al modelo clásico de OOP, basado en clases y herencia. Diseñamos e implementamos una serie de entornos iniciales para soportar este modelo que utilizamos en la primer parte del curso. Estos entornos eran adaptaciones del lenguaje y la herramienta principal de edición de Smalltalk, que es el lenguaje que usábamos para el resto del curso. Nos fue rebien (este párrafo está en la intro actual).
Previous works from our team \cite{lombardi_instances_2007,lombardi_carlos_alumnos_2008,griggio_programming_2011,spigariol_lucas_ensenando_2013} have described 
a novel path to introduce OO concepts focusing first on objects, messages and polymorphism while delaying the introduction of classes and inheritance%
.
A reduced and graphical programming environment supports this path, allowing to build OO programs without the need of classes.
This environment is an adaptation of the syntax and class browser of Pharo Smalltalk. After introducing classes, the course switched to the standard Pharo development tools.
 %and 
%(b) a reduced and graphical programming environment which supports the order in which we introduce the concepts, by allowing to build OO programs without the need of classes.
%Our approach focuses on the concepts of object, message, reference and object polymorphism, while delaying the introduction of more abstract concepts such as types, classes and inheritance.

This way of organizing a course provides a more gentle learning curve to students, so that they can write completely working programs from the first course weeks.

\medskip
%- A pesar del éxito, los entornos que implementamos no nos resultaron satisfactorios en algunos aspectos entre los que mencionamos: 
While this approach proved to be successful in providing the students with a more profound knowledge of OOP at the same time as raising pass rates, 
we feel that there is still room for improvement in three areas:
%  a. la gran distancia entre la experiencia en la materia y la realidad de la industria, tanto en el lenguaje como en las herramientas de desarrollo utilizadas
(a) the difference between the experience in the classroom and the reality in (most) professional environments, both in the language as in the development tools.
%  b. el salto entre el modelo simplificado y el modelo clásico que se daba en el curso, principalmente por la diferencia en las herramientas de desarrollo.
(b) the gap between the simplified and the classical programming models, mostly because of the differences among the development tools
%  c. atarnos a un lenguaje industrial, que nos limita para definir con precisión las herramientas pedagógicas.
(c) the adherence to a preexistent, general-purpose language that limited pedagogical decisions.

% What our solution is \ct{Set} and \ct{OrderedCollection} (so that the reader knows where the paper is going)
% Contribution of the paper
% (b) lo que aprendimos en estos 8 años hxaciendo eso, que nos lleva a querer darle una vuelta más.
As a result, we decided to conceive both a programming language and an accompanying development environment that  closely follow the pedagogical approach we advocate for an initial OOP course. 
The main goal of this paper is to describe Wollok\footnote{
	\url{http://www.wollok.org/}. 
	Source code and documentation can be found in Github 
	(\url{https://github.com/uqbar-project/wollok}).
	Wollok is open-source and distributed under LGPLv3 License 
	(\url{http://www.gnu.org/copyleft/lgpl.html}).}, 
the tool we created that reunites language and environment. 
%We point out the pedagogical considerations that led us to build Wollok, and how they influence its design decisions. We also report briefly our two-year experience using this tool in initial OOP courses.

\medskip
%%\section{Related Work}
%\label{sec:related}

% Other solutions in the domain, and a real comparison of our contribution with solutions from other people.
BlueJ \cite{bennedsen_bluej_2010} is an educative environment for programming in Java 
which shares several points of view with our approach (\cf \secref{related}).




Other approaches have put their focus in the 
A step further is to provide a whole programming environment specifically designed to aid novice programmers.

 
such as Squeak \cite{ingalls_back_1997}, 
Loop \cite{griggio_programming_2011}, 
and BlueJ .

Other environments make use of block-based or visual programming, 
such as Scratch \cite{malan_scratch_2007}, Etoys \cite{lee_empowering_2011} and Kodu \cite{kodu}. 
In our vision, these tools are suitable for stimulating interest in programming and for being used in secondary education, but not beyond that stage.

Other educators propose to use industrial languages in introductory courses, 
such as Java \cite{kolling2001guidelines}, Eiffel \cite{meyer1993towards}, Smalltalk \cite{ducasse2006squeak} and Self \cite{Unga87a}.


\section{Related Works}
\label{sec:related}

% Primero hablar de lenguajes específicos para enseñar
There have been proposals to tackle the first two problems by defining specific languages
that provide simplified programming models.
such as Karel++~\cite{bergin_karel++:_1996} and Mama~\cite{harrisonmama}.
This approach has been used even outside the OO world \cite{feurzeig_programming-languages_1970, pattis_karel_1981, lopez_nombre_2012}.
In this paper we use based on \emph{Wollok} \cite{passerini2017wollok}, 
an educative OO language designed to support a novel path to introduce OO concepts \cite{lombardi_instances_2007,lombardi_carlos_alumnos_2008,spigariol_lucas_ensenando_2013}.
This alternative learning path proposes to focus first on objects, messages an polymorphism, 
while delaying the introduction of more abstract concepts, such as classes, types or inheritance.

This approach has been used even outside the OO world \cite{feurzeig_programming-languages_1970, pattis_karel_1981, lopez_nombre_2012}.

\cite{passerini2017wollok}, 
which consists on a novel path to introduce OO concepts focusing first on objects, messages and polymorphism 

that supports an iterative learning path allows for a extremely simple initial programming model, 
as well as smooth transitions to a complete OO dynamicaly typed language.

which consists on a novel path to introduce OO concepts focusing first on objects, messages and polymorphism 

that supports an iterative learning path allows for a extremely simple initial programming model, 
as well as smooth transitions to a complete OO dynamicaly typed language.

% Environments
Still, the language by itself can not
A step further is to provide a whole programming environment specifically designed to aid novice programmers 
such as Squeak \cite{ingalls_back_1997}, 
Traffic \cite{broy_outside-method_2003},
Loop \cite{griggio_programming_2011}, 
and BlueJ \cite{bennedsen_bluej_2010}.

Other environments make use of block-based or visual programming, 
such as Scratch \cite{malan_scratch_2007}, Etoys \cite{lee_empowering_2011} and Kodu \cite{kodu}. 
In our vision, these tools are suitable for stimulating interest in programming and for being used in secondary education, but not beyond that stage.

Other educators propose to use industrial languages in introductory courses, 
such as Java \cite{kolling2001guidelines}, Eiffel \cite{meyer1993towards}, Smalltalk \cite{ducasse2006squeak} and Self \cite{Unga87a}.
%Self, at the same time, has pioneered in allowing for OOP without classes.


%There have been several proposals to address the difficulties in introductory OO courses 
%by defining a specific language which provides a simplified programming model such as Karel++~\cite{bergin_karel++:_1996} and Mama~\cite{harrisonmama}.
%This approach has been used even outside the OO world \cite{feurzeig_programming-languages_1970, pattis_karel_1981, lopez_nombre_2012}.
%% Environments
%A step further is to provide a whole programming environment specifically designed to aid novice programmers 
%such as Squeak \cite{ingalls_back_1997}, Traffic \cite{broy_outside-method_2003} and BlueJ \cite{bennedsen_bluej_2010}. 


Wollok combines some characteristics that are typically present in academical environments with others that are more easily seen in industrial ones%
%, aiming to enrich them to fulfill the specific needs of novice programmers and the proper development of a first OOP course
.
Its most noticeable characteristics are:
(1) the possibility of using self-defined objects to support an introduction to the topic that postpones the introduction of the concept of class,
(2) the possibility of combining self-defined with class-based code objects in the same program,
(3) the decission of offering an IDE with edition and code management capabilities that are fine-tuned for unexperienced programmers
and (4) that both the language as the IDE share similarities with their industrial counterparts, in order to soften the later transition to the professional tools.

\medskip 
In this short paper, we introduce our motivations for the development of Wollok in \secref{problem}, then we describe the Wollok language in \secref{wollokLanguage} and give some insights about its IDE in \secref{environment}. Finally, in \secref{experience-perspectives} we give a brief account of the experience using Wollok, and its perspectives.
% Paper structure
%%The rest of the paper is structured as follows. 
%In \secref{problem} we present the problems of learning Object Oriented programming, and the consequences of this difficulties to the students. \secref{wollokLanguage} describes the proposed language and design goals. 
%In \secref{environment} we describe the integrated development environment we have developed for Wollok and all the features it has and how they are useful for the teaching of programming skills. \secref{discussion} analyses the different design decisions we have taken.
%%, while \secref{related} compares our solution with another similar approaches. 
%Finally, we summarize our contributions in \secref{conclusion},
%along with some possible lines of further work derived from this initial ideas. 
%%As an appendix we have Sections \ref{sec:implementation}, \ref{sec:wollokSoftware} and \ref{sec:ChecksAndValidations} which give more details about the implementation.


\IEEEpubidadjcol